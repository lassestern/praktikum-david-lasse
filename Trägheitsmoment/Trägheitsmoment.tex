\documentclass[titlepage = firstcover]{scrartcl}
\usepackage[aux]{rerunfilecheck}
\usepackage{fontspec}
\usepackage[main=ngerman, english, french]{babel}

% mehr Pakete hier
\usepackage{expl3}
\usepackage{xparse}

%Mathematik------------------------------------------------------
\usepackage{amsmath}   % unverzichtbare Mathe-Befehle
\usepackage{amssymb}   % viele Mathe-Symbole
\usepackage{mathtools} % Erweiterungen für amsmath
\usepackage[
  math-style=ISO,    % \
  bold-style=ISO,    % |
  sans-style=italic, % | ISO-Standard folgen
  nabla=upright,     % |
  partial=upright,   % /
]{unicode-math}% "Does exactly what it says on the tin."

% Laden von OTF-Mathefonts
% Ermöglich Unicode Eingabe von Zeichen: α statt \alpha

\setmathfont{Latin Modern Math}
%\setmathfont{Tex Gyre Pagella Math} % alternativ zu Latin Modern Math
\setmathfont{XITS Math}[range={scr, bfscr}]
\setmathfont{XITS Math}[range={cal, bfcal}, StylisticSet=1]

\AtBeginDocument{ % wird bei \begin{document}
  % werden sonst wieder von unicode-math überschrieben
  \RenewDocumentCommand \Re {} {\operatorname{Re}}
  \RenewDocumentCommand \Im {} {\operatorname{Im}}
}
\usepackage{mleftright}
\setlength{\delimitershortfall}{-1sp}

%Sprache----------------------------------------------------------
\usepackage{microtype}
\usepackage{xfrac}
\usepackage[autostyle]{csquotes}    % babel
\usepackage[unicode, pdfusetitle]{hyperref}
\usepackage{bookmark}
\usepackage[shortcuts]{extdash}
%Einstellungen hier, z.B. Fonts
\usepackage{booktabs} % Tabellen

%Defininierte funktionen
\DeclareMathOperator{\f}{xyz}

\ExplSyntaxOn % bequeme Syntax für Definition von Befehlen

\NewDocumentCommand \I {} {         %Befehl \I definieren,keine Argumente
  \symup{i}                         %Ergebnis von \I
} 
\NewDocumentCommand \dif {m} % m = mandatory (Pflichtargument für \dif)
{
  \mathinner{\symup{d} #1}
}

\ExplSyntaxOff % Syntax wieder ausschalten. Wichtig!


\title{Das Trägheitsmoment}
\author{David Gutniko \and Lasse Sternemann emails}
\date{Durchführung am 19.11.2019}

\begin{document}
    \maketitle

    \section{Zielsetzung}
    Es sollen Trägheitsmomente verschiedener Körper durch Experimente gemessen und durch entsprechende Rechnungen überprüft werden. 
    Dabei wird auch der in den Rechnungen angewandte Satz von Steiner durch die Messungen bestätigt. Die verschiedenen Körper sind dabei 
    allesamt Symmetriekörper wie Kugeln und Zylinder oder werden als diese angenähert.

    \section{Theoretische Grundlagen}
    Das Trägheitsmoment beschreibt das Streben einer Masse sich gegen die Änderung der Winkelgeschwindigkeit zu widersetzen. Damit stellt es 
    das Äquivalent zur trägen Masse dar, welche sich der translativen Beschleunigung entgegensetzt (!).
    Standardmäßig betrachtet man das Trägheitsmoment bezüglich der Drehachse durch den Schwerpunkt des Körpers. So lässt sich das Trägheitsmoment 
    $j$ für einen Körper aus vielen homogen verteilten Massenpunkten m durch Formel 1 berechnen. Dabei bezeichnet $m$ die Masse der einzelnen 
    Massenpunkte und $r$ deren senkrechten Abstand zur Drehachse.
    \begin{equation}
      J = \sum_(i=1)^n m_i * r_i^2
    \end{equation}
    Um die Berechnung auf für reelle Körper genau möglich zu machen, reduziert man die Größe der Massenpunkte bis sie infinitesimal klein werden und man 
    nun anstatt aufzusummieren über sie Integrieren kann. Daraus ergibt sich Formel 2.
    \begin{equation}
      J = \int r^2 dm
    \end{equation}
    Da die Masse der einzelnen Massenpunkte nicht bekannt ist, werden Trägheitsmomente normalerweise berechnet, indem man die Dichteverteilung des Körpers 
    über das Volumen integriert. Bei den gemessenen Körpern liegen homogene Massenverteilungen vor, sodass sich zur Berechnung derer Trägheitsmomente 
    Formel 2 ergibt.  



\end{document}
