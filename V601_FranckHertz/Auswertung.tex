\documentclass[titlepage = firstcover]{scrartcl}
\usepackage[aux]{rerunfilecheck}
\usepackage{fontspec}
\usepackage[main=ngerman, english, french]{babel}

% mehr Pakete hier
\usepackage{expl3}
\usepackage{xparse}

%Mathematik------------------------------------------------------
\usepackage{amsmath}   % unverzichtbare Mathe-Befehle
\usepackage{amssymb}   % viele Mathe-Symbole
\usepackage{mathtools} % Erweiterungen für amsmath
\usepackage{dsfont}
\usepackage[
  math-style=ISO,    % \
  bold-style=ISO,    % |
  sans-style=italic, % | ISO-Standard folgen
  nabla=upright,     % |
  partial=upright,   % /
]{unicode-math}% "Does exactly what it says on the tin."
\usepackage[section, below]{placeins}

% Laden von OTF-Mathefonts
% Ermöglich Unicode Eingabe von Zeichen: α statt \alpha

\setmathfont{Latin Modern Math}
%\setmathfont{Tex Gyre Pagella Math} % alternativ zu Latin Modern Math
\setmathfont{XITS Math}[range={scr, bfscr}]
\setmathfont{XITS Math}[range={cal, bfcal}, StylisticSet=1]

\AtBeginDocument{ % wird bei \begin{document}
  % werden sonst wieder von unicode-math überschrieben
  \RenewDocumentCommand \Re {} {\operatorname{Re}}
  \RenewDocumentCommand \Im {} {\operatorname{Im}}
}
\usepackage{mleftright}
\setlength{\delimitershortfall}{-1sp}
\usepackage[version=4]{mhchem}

%Sprache----------------------------------------------------------
\usepackage{microtype}
\usepackage{xfrac}
\usepackage[autostyle]{csquotes}    % babel
\usepackage[german, unicode, pdfusetitle]{hyperref}
\usepackage{bookmark}
\usepackage[shortcuts]{extdash}
%Einstellungen hier, z.B. Fonts
\usepackage{booktabs} % Tabellen

\setlength{\parindent}{0pt}

\title{V601 - Der Franck-Hertz Versuch}
\author{
  David Gutnikov\\
  \href{mailto:david.gutnikov@udo.edu}{david.gutnikov@udo.edu}\\
  Lasse Sternemann\\
  \href{mailto:lasse.sternemann@udo.edu}{lasse.sternemann@udo.edu}
}
\date{Bearbeitet am 7.07.2020}

\begin{document}
    \maketitle
    \newpage
    \tableofcontents
    \newpage

    \section{Auswertung}    
        \subsection{Bestimmung der mittleren freien Weglänge}
            Zunächst wird überprüft, ob die freie Weglänge $\overline{w}$ die Bedingung erfüllt, um ca. das tausendfache kleiner als der Abstand a zwischen Glühkathode und Beschleunigungsanode zu sein.
            Die freien Weglängen berechnen sich über Formel xx und werden daraufhin zur Berechnung des Verhältnis zwischen a=1cm und $\overline{w}$ genutzt.

            \begin{table}[h]
                \centering
                \caption{In der Tabelle sind die den Temperaturen zugehörigen mittleren Weglängen der Gasatome, sowie das zugehörige Verhältnis zur Beschleunigungsdistanz a.}
                \label{tab:TabGasdruck}

                \begin{tabular}{c  c c}
                    \toprule
                    {$T \; [\text{K}] $} & {$\overline{w} \; [\text{m}\cdot 10^{-6}]$} &  {$\frac{\text{a}}{\overline{w}}$} \\
                    \midrule
                    300,85 & 4500 & 2,3 \\
                    424,15 & 5,8 &  1800  \\
                    443,15 & 2,9 &  3500  \\
                    470,25 & 1,2 &  8500  \\
                    \bottomrule
                \end{tabular}

            \end{table}
        
        \subsection{Bestimmung der Skalenanteile}
            Zunächst wird berechnet wie viel Volt ein Kästchen auf dem Graphen der jeweiligen Temperatur des xy-Schreibers entspricht. Dazu wird folgende Formel benutzt, die zunächst aus den 
            einzelnen Abständen $d_i$ den durchschnittlichen Abstand zwischen den 1 Volt Markierungen bestimmt und daraus die Volt pro Kästchen.

            \begin{equation*}
                \text{Volt/Kästchen} = \sum_{i=1}^{n} \frac{1}{d_i}
            \end{equation*}

            \begin{table}[h]
                \centering
                \caption{I die Masse des Wassers im Dewar-Gefäß $m_{\text{w}}$ angegeben.}
                \label{tab:Tabelle1}

                \begin{tabular}{c c c c c}
                    \toprule
                    {} & {300,85 K} & {424,15 K} \\
                    \midrule
                    Kästchen/Volt & 24 & 24 \\
                                  & 24 & 24 \\
                                  & 24 & 25 \\
                                  & 25 & 24 \\
                                  & 24 & 26 \\
                                  & 24 & 25 \\
                                  & 25 & 25 \\
                                  & 26 & 25 \\
                                  & 26 & 27 \\
                                  & 28 & 24 \\
                                  Graphit       & 80,0 & 21,4 & 23,2 & 0,56694 \\
                    \bottomrule
                \end{tabular}

            \end{table}
            
            
            
            


        \end{document}