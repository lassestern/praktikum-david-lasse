\documentclass[titlepage = firstcover]{scrartcl}
\usepackage[aux]{rerunfilecheck}
\usepackage{fontspec}
\usepackage[main=ngerman, english, french]{babel}

% mehr Pakete hier
\usepackage{expl3}
\usepackage{xparse}

%Mathematik------------------------------------------------------
\usepackage{amsmath}   % unverzichtbare Mathe-Befehle
\usepackage{amssymb}   % viele Mathe-Symbole
\usepackage{mathtools} % Erweiterungen für amsmath
\usepackage[
  math-style=ISO,    % \
  bold-style=ISO,    % |
  sans-style=italic, % | ISO-Standard folgen
  nabla=upright,     % |
  partial=upright,   % /
]{unicode-math}% "Does exactly what it says on the tin."

% Laden von OTF-Mathefonts
% Ermöglich Unicode Eingabe von Zeichen: α statt \alpha

\setmathfont{Latin Modern Math}
%\setmathfont{Tex Gyre Pagella Math} % alternativ zu Latin Modern Math
\setmathfont{XITS Math}[range={scr, bfscr}]
\setmathfont{XITS Math}[range={cal, bfcal}, StylisticSet=1]

\AtBeginDocument{ % wird bei \begin{document}
  % werden sonst wieder von unicode-math überschrieben
  \RenewDocumentCommand \Re {} {\operatorname{Re}}
  \RenewDocumentCommand \Im {} {\operatorname{Im}}
}
\usepackage{mleftright}
\setlength{\delimitershortfall}{-1sp}

%Sprache----------------------------------------------------------
\usepackage{microtype}
\usepackage{xfrac}
\usepackage[autostyle]{csquotes}    % babel
\usepackage[unicode, pdfusetitle]{hyperref}
\usepackage{bookmark}
\usepackage[shortcuts]{extdash}
%Einstellungen hier, z.B. Fonts
\usepackage{booktabs} % Tabellen


\title{Das Dulong-Petitsche-Gesetz}
\author{
  David Gutnikov\\
  \href{mailto:david.gutnikov@udo.edu}{david.gutnikov@udo.edu}
 \and 
  Lasse Sternemann\\
  \href{mailto:lasse.sternemann@udo.edu}{lasse.sternemann@udo.edu}
}
\date{Durchführung am 03.12.2019}


\begin{document}
    \maketitle
    \newpage
    \tableofcontents
    \newpage

    \section{Zielsetzung}
        Es soll überprüft werden, ob das Dulong-Petit-Gesetz bei genügend hohen Temperaturen treffende Aussagen zur Molwärme von Feststoffen trifft.



    \section{Theoretische Grundlagen}
        \subsection{Wärmekapazität}
            Wenn ein Stoff, nur erhitzt wird und im Prozess keine andere Energie aufnimmt oder abgibt, nimmt er die
            Wärmemenge $\Delta Q$ auf, d.h. es gilt:
            \begin{equation*}
                \Delta U = \Delta Q + \Delta A = \Delta Q
            \end{equation*}
            $U$ und $E$ stehen im Protokoll für gemittelte Energien.
            Die aufgenommene Wärmemenge $\Delta Q$ hängt dabei von der Änderung der Temperatur $\Delta T$,
            der Masse des Körpers $m$, sowie der spezifischen Wärmekapazität des Elements $c$ ab. Dieser Zusammenhang
            definiert die spezifische Wärmekapazität eines Stoffes, wie in Formel \eqref{eqn:wk} beschrieben.  
            \begin{equation}
                c = \frac{\Delta Q}{\Delta T \cdot m }
                \label{eqn:wk}
            \end{equation}
        
        \subsection{Molwärme nach dem Dulong-Petitschen-Gesetz}
            Das Dulong-Petitsche-Gesetz geht davon aus, dass alle Festkörper die gleiche Molwärme inne haben. Dies wird aus der
            Teilchenenergie, der im Körper gebundenen Atome hergeleitet, indem davon ausgegangen wird, dass sie sich in einer festen Position
            innerhalb einer Gitterstruktur befinden. Ihre einzig mögliche Bewegung ist eine harmonische Schwingung um ihre Ruheposition. Wie für
            harmonische Oszilation üblich ist die gemittelte kinetische Energie der Teilchen gleich der gemittelten potentiellen Energie, sodass sich für die gesamte innere Energie eines 
            Teilchens im Festkörper pro Freiheitsgrad Formel \eqref{eqn:eteilchen} ergibt, in der $k_{\text{b}}$ für die Boltzmann-Konstante steht.
            \begin{equation}
                U_{\text{Teilchen pro Freiheitsgrad}} = 2 \cdot E_{\text{Kin}} = k_{\text{b}} \cdot T \quad \xrightarrow[\text{3 Freiheitsgrade}]{} \quad U_{\text{Teilchen}} = 3 \cdot k_{\text{b}} \cdot T
                \label{eqn:eteilchen}
            \end{equation} 
            Da der Festkörper natürlich aus mehreren Teilchen besteht, wird die gesamte innere Energie eines Mols der Teilchen aus denen der Festkörper besteht durch Formel \eqref{eqn:ekörper} beschrieben, 
            wobei $N_{\text{A}}$ die Avogrado-Konstante ist, die multipliziert mit der Boltzmann-Konstante die allgemeine Gaskonstante $R$ ergibt. 
            \begin{equation}
                U_{\text{Körper}} = 3 \cdot N_{\text{A}} \cdot k_{\text{b}} \cdot T = 3 \cdot R \cdot T := Q
                \label{eqn:ekörper}
            \end{equation}
            Über diese Energie lässt sich die spezifische Molwärme beschreiben, die beschreibt wie viel Wärmemenge hinzugefügt werden muss, um ein Mol eines
            Elements um eine Temperatur $\Delta T$ zu erhitzen, sofern der Druck konstant ist. Es ergibt sich für alle Festkörper aus einem Element
            \begin{equation}
                C_{\text{v}} = \frac{dU}{dT} = 3R \qquad \text{mit} \qquad dU = dQ + dA = dQ .
            \end{equation}
            
        \subsection{Einschränkung des Dulong-Petitschen-Gesetz}
            Während der über das Dulong-Petit-Gesetz gegebene Wert von 3R bei hohen Temperaturen von ca. 20°C und höher gut mit den tatsächlichen Werten übereinstimmen soll,
            fallen die Werte für niedrige Temperaturen stark ab. Dies liegt an der Verquantelung der harmonischen Schwingungen in der Gitterstruktur. 
            Durch diese können die einzelnen Energien nur ganzzahlige Vielfache des Produkts der Schwingfrequenz und des Planckschen Wirkungsquantum annehmen. Wenn 
            die Schwingungen bei kleinen Temperaturen immer kleiner werden, kann es passieren, dass die Schwingungen so gering werden, dass ihre Energie
            kleiner als das $1 \cdot \hbar \omega$ ist und sie garkeinen Beitrag mehr liefern, sodass die Molwärme gegen 0 geht. Dieses Verhalten kann man aus Formel
            \eqref{eqn:QU} entnehmen, die dem gerade beschriebenen quantenmechanischen Ansatz entspricht und die innere Energie eines schwingenden Teilchens beschreibt.
            \begin{equation}
              U_{\text{Teilchen}} = \frac{3 N_{\text{A}} \hbar \omega}{e^{\frac{\hbar \omega}{k_bT}}-1}
              \label{eqn:QU}
            \end{equation} 
            Der Exponentialterm geht für $T \rightarrow 0$ gegen $\infty$ und die Molwärme dementsprechend gegen 0. Um die Molwärme für Große Temperaturen
            zu betrachten wird die Taylor-Entwicklung des Exponentialterms \eqref{eqn:TEU} genutzt. Mit dieser geht die Molwärme für hohe Temperaturen gegen 
            den klassischen Wert von $3R$.
            \begin{equation}
              U_{\text{Teilchen}} \approx \frac{3 N_{\text{A}} \hbar \omega}{1 + \frac{\hbar \omega}{k_{\text{b}T}}}
              \label{eqn:TEU}
            \end{equation} 

    \newpage

    \section{Versuchsdurchführung}
        Das konstante Volumen einer Körperprobe ist bei Durchführung eines Versuches schwierig zu erhalten.
        Deshalb wird hier mit einer anderen konstanten Größe gearbeitet, dem konstanten Druck, welcher hier
        der Atmosphärendruck ist. Es werden Körper aus verschiedenen Materialien der Massen $m_\text{k}$ in Wasser
        eingetaucht und das Wasser wird auf die Temperatur $T_\text{k}$ erhitzt. Dabei nimmt der Körper die Temperatur $T_\text{k}$ an. Ab einer genügend hohen Temperatur
        wird der Körper aus dem Gefäß entfernt und in ein mit Wasser mit Zimmertemperatur $T_\text{w}$ gefülltes
        Dewar-Gefäß getaucht. Nach ca. 2 Minuten des Wartens ergibt sich eine Mischtemperatur $T_\text{m}$ im Gefäß.
        Die verschiedenen Temperaturen werden mit einem Thermometer gemessen.
        Die spezifischen Wärmekapazitäten werden dann mit der spezifischen Wärmekapazität des Körpers $c_\text{k}$, die des Wassers im Dewar-Gefäß bei ca. Zimmertemperatur
        $c_\text{w}$, der Wärmekapazität des Dewar-Gefäßes $c_\text{g}m_\text{g}$ und der Masse des Wasser im Dewar-Gefäß
        $m_\text{w}$ nach \eqref{eqn:spezWärmekapazität} bestimmt.
        
        \noindent
        Um die Wärmekapazität des Dewar-Gefäßes zu bestimmen ist eine weitere Messung nötig. Anstatt einer Körperprobe
        wird nun Wasser der Masse $m_\text{y}$ auf eine Temperatur $T_\text{y}$ erhitzt und es wird damit genauso verfahren wie mit den Körpern.
        D.h. es wird in das Dewar-Gefäß gefüllt, worin sich schon Wasser der Masse $m_\text{x}$ und der Temperatur
        $T_\text{x}$ befindet und mit welchem sich eine Mischtemperatur $T_\text{m}^{'}$ ergibt. Analog zu
        \eqref{eqn:spezWärmekapazität} gilt für die Wärmekapazität des Dewar-Gefäßes \eqref{eqn:Wärmekapazität}.
    

    \section{Auswertung}
        \subsection{Spezifische Wärmekapazität bei konstantem Druck}
            Die Körperprobe gibt beim Eintauchen in das Wasser im Dewar-Gefäß die Wärmemenge $Q_1$
            \begin{equation*}
                Q_1 = c_\text{k}m_\text{k}(T_\text{k} - T_\text{m})
            \end{equation*}
            ab, wobei das Wasser und das Dewar-Gefäß dieselbe Wärmemenge $Q_2$ aufnehmen.
            \begin{equation*}
                Q_2 = (c_\text{w}m_\text{w} + c_\text{g}m_\text{g})(T_\text{k} - T_\text{m})
            \end{equation*}
            Demnach können $Q_1$ und $Q_2$ gleichgesetzt und nach der spezifischen Wärmekapazität $c_k$ umgestellt werden. Daraus folgt \eqref{eqn:spezWärmekapazität}.
            
            \begin{equation}
                \label{eqn:spezWärmekapazität}
                c_\text{k} = \frac{(c_\text{w} m_\text{w} + c_\text{g} m_\text{g})(T_\text{m} - T_\text{w})}{m_\text{k}(T_\text{k} - T_\text{m})}
            \end{equation}
            Die letzte noch fehlende Größe in \eqref{eqn:spezWärmekapazität} ist die Wärmekapazität des Dewar-Gefäßes $c_\text{g}m_\text{g}$. Diese
            wird analog zu $c_{\text{k}}$ bestimmt, sodass sich folgende Formel ergibt:

            \begin{equation}
                \label{eqn:Wärmekapazität}
                c_\text{g} m_\text{g} = \frac{c_\text{w} m_\text{y}(T_\text{y} - T_\text{m}^{'}) - c_\text{w} m_\text{x}(T_\text{m}^{'} - T_\text{x})}{(T_\text{m}^{'} - T_\text{x})}
            \end{equation}
        
        

        \subsection{Spezifische Wärmekapazität verschiedener Feststoffe}
            Um \eqref{eqn:spezWärmekapazität} nutzen zu können, wurde die Wärmekapazität des Dewar-Gefäßes bestimmt, wie in der Versuchsdurchführung
            beschrieben. Dabei wurden die Werte 
            \begin{align*}
                m_{\text{y}} &= 0,28971 \:\text{kg} \\
                T_{\text{y}} &= 67,5 \:\text{°C} \\
                m_{\text{x}} &= 0,26010 \: \text{kg} \\
                T_{\text{x}} &= 21,3 \:\text{°C} \\
                T_{\text{m}} &= 43,1 \:\text{°C} 
            \end{align*}
            verwendet, sodass $c_\text{g} m_\text{g} = 268,6 \: \frac{\text{J}}{\text{K}}$. Mit diesem Wert lassen sich nun die spezifischen Wärmekapazitäten
            bei konstantem Druck berechnen.
            \begin{table}[h]
                \centering
                \caption{In der Tabelle sind die Temperaturen der erhitzten Körper $T_{\text{k}}$, die des Wasser im Dewar-Gefäß bevor der erhitzte Körper hineingelassen wird $T_{\text{w}}$ und nachdem der Körper hineingelassen wurde $T_{\text{m}}$. Zuletzt ist noch die Masse des Wassers im Dewar-Gefäß $m_{\text{w}}$ angegeben.}
                \label{tab:Tabelle1}

                \begin{tabular}{c c c c c}
                    \toprule
                    {Stoff} & {$T_{\text{k}} \; [\text{°C}] $} & {$T_{\text{w}} \; [\text{°C}]$} &  {$T_{\text{m}} \; [\text{°C}]$} & {$m_{\text{w}} \; [\text{kg}]$}\\
                    \midrule
                    Aluminium & 76,0 & 21,9 & 23,1 & 0,56288 \\
                              & 99,4 & 22,4 & 26,6 & 0,56416 \\
                              & 73,3 & 21,9 & 22,9 & 0,56078 \\
                    Kupfer    & 78,1 & 21,7 & 24,7 & 0,56803 \\
                              & 78,3 & 21,4 & 24,5 & 0,57302 \\
                              & 77,4 & 21,5 & 24,4 & 0,58335 \\
                    Graphit   & 80,0 & 21,4 & 23,2 & 0,56694 \\
                    \bottomrule
                \end{tabular}

            \end{table}

            \noindent
            Zuerst werden die $c_{\text{k}}$ für Aluminium und Kupfer gemittelt (es gab nur eine Messung zu Graphit, deshalb kann da nicht gemittelt werden)
            nach \eqref{eqn:mittelnck}
            \begin{equation*}
                \overline{c_{\text{k}}} = \frac{1}{n} \cdot \sum_{n=1}^3 c_{\text{k}_n} \qquad \text{mit} \quad n=3
            \end{equation*}
            Aus den Werten von \ref{tab:Tabelle1} und den Massen der zu erhitzenden Körper
            \begin{align*}
                m_{\text{Al}} =& 0,11279 \; \text{kg} \\
                m_{\text{Cu}} =& 0,23645 \; \text{kg} \\
                m_{\text{Graphit}} =& 0,10760 \; \text{kg}
            \end{align*}
            werden nun die $\overline{c_{\text{k}}}$ zu den einzelnen Metallen berechnet, indem von 
            Formel \eqref{eqn:spezWärmekapazität} gebrauch gemacht wird. Diese Werte werden über
            \begin{equation*}
                C_{\text{p}} = \frac{\overline{c_{\text{k}}} \cdot m_{\text{k}}}{n_{\text{k}}} = \overline{c_{\text{k}}} \cdot M
            \end{equation*} 
            zu $C_{\text{p}}$ umgerechnet, wobei M [1, S.8] für das Verhältnis von Masse und Stoffmenge steht und auf der Praktikumsanleitung angegeben ist. 
            Die gesuchten $C_{\text{V}}$ werden nun folgendermaßen berechnet.
            \begin{align}
                C_{\text{V}} &= C_{\text{p}} \hspace*{-3.5cm}&-\quad 9 \alpha^2 \upkappa  V_0 T_{\text{m}} \\
                \label{eqn:CV} 
                             &= \overline{c_{\text{k}}}\cdot M \hspace*{-3.5cm}&-\quad 9 \alpha^2 \upkappa  V_0 T_{\text{m}}
            \end{align}
            
            \begin{table}[h]
                \centering
                \caption{Für die einzelnen Stoffe wurden die molaren Wärmekapazitäten bei konstantem Druck $C_{\text{p}}$ und konstantem Volumen $C_{\text{V}}$ aus der spezifischen Wärmekapazität $c_{\text{k}}$ berechnet.}
                \label{tab:Tabelle2}

                \begin{tabular}{c c c c c}
                    \toprule
                    {Stoff} & {$M \; [{\frac{\text{kg}}{\text{mol}}}] $} & {$c_{\text{k}} \; [{\frac{\text{J}}{\text{kgK}}}] $} & {$C_{\text{p}} \; [{\frac{\text{J}}{\text{molK}}}]$} &  {$C_{\text{V}} \; [{\frac{\text{J}}{\text{molK}}}]$} \\
                    \midrule
                    Aluminium & 0,0270 & 777,8 & 21,0 & 19,9 \\
                    Kupfer    & 0,0635 & 635,3 & 40,3 & 39,6 \\
                    Graphit   & 0,0120 & 778,1 & 9,3 & 9,3 \\
                    \bottomrule
                \end{tabular}

            \end{table}
            
            \begin{equation*}
                \Delta C_{\text{V}} = \sqrt{\Delta C_p^2 + (9 \alpha^2 \upkappa  V_0)^2 \cdot \Delta T_{\text{m}}^2}
            \end{equation*}

            \begin{align*}
                \text{Aluminium} &: C_{\text{V}} =& (20 \pm 8) \frac{\text{J}}{\text{mol \cdot K}} \\
                \text{Kupfer}    &: C_{\text{V}} =& (39 \pm 1) \frac{\text{J}}{\text{mol \cdot K}} \\
                \text{Graphit}   &: C_{\text{V}} =& 9 \frac{\text{J}}{\text{mol \cdot K}} 
            \end{align*}
            Der Literaturwert [2] ist nach dem klassischen Dulong-Petit-Gesetz $c_{\text{k}} = 3R = 24,9 \frac{\text{J}}{\text{mol \cdot K}}$.
            Daraus ergeben sich folgende Abweichungen vom Literaturwert [2]:
            \begin{align*}
                \text{Aluminium} &: 19,7 \%  \\
                \text{Kupfer}    &: 56,6 \%   \\
                \text{Graphit}   &: 63,9 \%
            \end{align*}
        
            \newpage
        
    \section{Diskussion}
        Die aus den Messwerten berechneten spezifischen Wärmekapazitäten liegt nur bei Aluminium relativ nah an dem aus dem klassischen Dulong-Petit-Gesetz
        stammenden Literaturwert [2] von 3R. Ansonsten sind beim Kupfer und Graphit sehr weit vom tatsächlichen Wert liegende Werte rausgekommen.
        Die größte Fehlerquelle liegt darin, dass der  erhitzte Körper definitiv nicht seine gesamte Wärmeenergie an das Wasser im Gefäß abgibt, da viel
        Wärme an die Umwelt abgegeben wird. Außerdem wurde wahrscheinlich eine oder mehrere Temperaturmessungen daruch verfälscht, dass das Thermometer
        zeitweise direkt mit dem Material in Berührung stand und teilweise nicht.
        Die zweite Fehlerquelle ist das Umfüllen vom Wasser in das Dewar-Gefäß, da dabei Wasser in Form von Dampf verloren geht, das jedoch nicht vom gemessen 
        Wassergewicht abgezogen worden ist. Weitere im Vergleich kleine Messfehler ergeben sich durch eine nicht perfekte Eichung der Waage und des
        Thermometers. Durch die gemessenen Ungenauigkeiten kann das Dulong-Petit-Gesetz nicht bestätigt werden.

    \newpage
    
    \section{Literaturverzeichnis}
        [1] \textit{Versuchanleitung V201 - Das Dulong-Petitsche-Gesetz.} TU Dortmund, 2019 \newline
        [2] National Institute of Standards and Technology: \textit{Fundamental Physical Constants} 09.Dezember.2019
            \url{https://physics.nist.gov/cgi-bin/cuu/Value?r}
        

\end{document}