\documentclass[titlepage = firstcover]{scrartcl}
\usepackage[aux]{rerunfilecheck}
\usepackage{fontspec}
\usepackage[main=ngerman, english, french]{babel}

% mehr Pakete hier
\usepackage{expl3}
\usepackage{xparse}

%Mathematik------------------------------------------------------
\usepackage{amsmath}   % unverzichtbare Mathe-Befehle
\usepackage{amssymb}   % viele Mathe-Symbole
\usepackage{mathtools} % Erweiterungen für amsmath
\usepackage[
  math-style=ISO,    % \
  bold-style=ISO,    % |
  sans-style=italic, % | ISO-Standard folgen
  nabla=upright,     % |
  partial=upright,   % /
]{unicode-math}% "Does exactly what it says on the tin."

% Laden von OTF-Mathefonts
% Ermöglich Unicode Eingabe von Zeichen: α statt \alpha

\setmathfont{Latin Modern Math}
%\setmathfont{Tex Gyre Pagella Math} % alternativ zu Latin Modern Math
\setmathfont{XITS Math}[range={scr, bfscr}]
\setmathfont{XITS Math}[range={cal, bfcal}, StylisticSet=1]

\AtBeginDocument{ % wird bei \begin{document}
  % werden sonst wieder von unicode-math überschrieben
  \RenewDocumentCommand \Re {} {\operatorname{Re}}
  \RenewDocumentCommand \Im {} {\operatorname{Im}}
}
\usepackage{mleftright}
\setlength{\delimitershortfall}{-1sp}

%Sprache----------------------------------------------------------
\usepackage{microtype}
\usepackage{xfrac}
\usepackage[autostyle]{csquotes}    % babel
\usepackage[unicode, pdfusetitle]{hyperref}
\usepackage{bookmark}
\usepackage[shortcuts]{extdash}
%Einstellungen hier, z.B. Fonts
\usepackage{booktabs} % Tabellen


\title{Das Dulong-Petitsche-Gesetz}
\author{
  David Gutnikov\\
  \href{mailto:david.gutnikov@udo.edu}{david.gutnikov@udo.edu}
 \and 
  Lasse Sternemann\\
  \href{mailto:lasse.sternemann@udo.edu}{lasse.sternemann@udo.edu}
}
\date{Durchführung am 03.12.2019}

\begin{document}
    \maketitle
    \tableofcontents
    \newpage

    \section{Zielsetzung}
        sss


    \section{Theoretische Grundlagen}
        \subsection{Wärmekapazität}
            Wenn ein Stoff, erhitzt wird und im Prozess keine andere Energie aufnimmt oder abgibt,
            \begin{equation*}
                \Delta U = \Delta Q + \Delta A = \Delta Q
            \end{equation*}
            nimmt er die Wärmemenge $\Delta Q$ auf. 
            Die aufgenommene Wärmemenge $\Delta Q$ hängt dabei von der Änderung der Temperatur $\Delta T$,
            der Masse des Körpers $m$, sowie der spezifischen Wärmekapazität des Elements $c$ ab. Dieser Zusammenhang definiert die spezifische
            Wärmekapazität eines Stoffes, wie in Formel \eqref{eqn:wk} beschrieben.  
            \begin{equation}
                c = \frac{\Delta Q}{\Delta T \cdot m }
                \label{eqn:wk}
            \end{equation}
        
        \subsection{Molwärme nach dem Dulong-Petitschen-Gesetz}
            Das Dulong-Petitsche-Gesetz geht davon aus, dass alle Festkörper die gleiche spezifische Wärmekapazität inne habe. Dies wird aus der
            Teilchenenergie, der im Körper gebunden Atome hergeleitet, indem davon ausgegangen wird, dass sie sich in einer festen Position
            innerhalb einer Gitterstruktur befinden. Ihre einzig mögliche Bewegung ist eine harmonische Schwingung um ihre Ruheposition. Wie für
            harmonische üblich ist die kinetische Energie der Teilchen gleich der potentiellen Energie, sodass sich für die gesamte innere Energie eines 
            Teilchens im Festkörper pro Freiheitsgrad Formel \eqref{eqn:eteilchen} ergibt, in der $k_{\text{b}}$ für die Boltzmann-Konstante steht.
            \begin{equation}
                U_{\text{Teilchen pro Freiheitsgrad}} = 2 \cdot E_{\text{Kin}} = k_{\text{b}} \cdot T \longrightarrow_{3 Freiheitsgrade} U_{\text{Teilchen}} = 3 \cdot k_{/text{b}} \cdot T
                \label{eqn:eteilchen}
            \end{equation} 
            Da der Festkörper natürlich aus mehreren Teilchen besteht, wird die gesamte innere Energie des Festkörpers durch Formel \eqref{eqn:ekörper} beschrieben, 
            wobei $N_{\text{A}}$ die Avogrado-Konstante ist, die multipliziert mit der Boltzmann-Konstante die allgemeine Gaskonstante $R$ ergibt. 
            \begin{equation}
                U_{\text{Körper}} = 3 \cdot N_{\text{A}} \cdot k_{\text{b}} = 3 \cdot R \cdot T := Q
                \label{eqn:ekörper}
            \end{equation}
            Über diese Energie lässt sich die spezifische Molwärme beschreiben, die beschreibt wie viel Wärmemenge hinzugefügt werden muss, um ein Mol eines
            Elements um eine Temperatur $\Delta T$ zu erhitzen, sofern der Druck konstant ist. Es ergibt sich für alle Festkörper aus einem Element
            \begin{equation}
                c_{\text{v}} = \frac{dU}{dT} = 3R \qquad \text{mit} \qquad dU = dQ + dA = dQ .
            \end{equation}
<<<<<<< HEAD
            
    \newpage

    \section{Versuchsdurchführung}
        Das konstante Volumen einer Körperprobe ist bei Durchführung eines Versuches schwierig zu erhalten.
        Deshalb arbeitet man hier mit einer anderen konstanten Größe, dem konstanten Druck, welcher hier
        der Atmosphärendruck ist. Es werden Körper aus verschiedenen Materialien der Massen $m_\text{k}$ in Wasser
        eingetaucht und das Wasser wird auf die Temperatur $T_\text{k}$ erhitzt. Dabei nimmt der Körper die Temperatur $T_\text{k}$ an. Ab einer genügend hohen Temperatur
        wird der Körper aus dem Gefäß entfernt und in ein mit Wasser mit Zimmertemperatur $T_\text{w}$ gefülltes
        Dewar-Gefäß getaucht. Nach ca. 2 Minuten des Wartens ergibt sich eine Mischtemperatur $T_\text{m}$ im Gefäß.
        Die verschiedenen Temperaturen werden mit einem Thermometer gemessen.
        Die spezifischen Wärmekapazitäten werden, dann mit der spezifischen Wärmekapazität des Körpers $c_\text{k}$, des Wassers im Dewar-Gefäß bei ca. Zimmertemperatur
        $c_\text{w}$, der Wärmekapazität des Dewar-Gefäßes $c_\text{g}m_\text{g}$ und der Masse des Wasser im Dewar-Gefäß
        $m_\text{w}$ nach Formel \eqref{eqn:spezWärmekapazität} bestimmt.

        Um die Wärmekapazität des Dewar-Gefäßes zu bestimmen ist ein weiteres Experiment nötig. Anstatt einer Körperprobe
        wird nun Wasser der Masse $m_\text{y}$ auf eine Temperatur $T_\text{y}$ erhitzt und es wird damit genauso verfahren wie mit den Körpern.
        D.h. es wird in das Dewar-Gefäß gegossen, worin sich schon Wasser der Masse $m_\text{x}$ und der Temperatur
        $T_\text{x}$ befindet und mit welchem sich eine Mischtemperatur $T_\text{m}^{'}$ ergibt. Analog zu Formel
        \eqref{eqn:spezWärmekapazität} gilt für die Wärmekapazität des Dewar-Gefäßesnach Formel \eqref{eqn:Wärmekapazität}.
    

    \section{Auswertung}
        \subsection{Spezifische Wärmekapazität bei konstentem Druck}
            Die Körperprobe gibt beim eintauchen in das Wasser im Dewar-Gefäß die Wärmemenge $Q_1$
            \begin{equation*}
                Q_1 = c_\text{k}m_\text{k}(T_\text{k} - T_\text{m})
            \end{equation*}
            ab, wobei das Wasser und das Dewar-Gefäß die Wärmemenge $Q_2$ aufnimmt.
            \begin{equation*}
                Q_2 = (c_\text{w}m_\text{w}) + c_\text{g}m_\text{g})(T_\text{k} - T_\text{m})
            \end{equation*}
            

            \begin{equation}
                \label{eqn:Wärmekapazität}
                c_\text{g} m_\text{g} = \frac{c_\text{w} m_\text{y}(T_\text{y} - T_\text{m}^{'}) - c_\text{w} m_\text{x}(T_\text{m}^{'} - T_\text{x})}{(T_\text{m}^{'} - T_\text{x})}
            \end{equation}

            \begin{equation}
                \label{eqn:spezWärmekapazität}
                c_\text{k} = \frac{(c_\text{w} m_\text{w} + c_\text{g} m_\text{g})(T_\text{m} - T_\text{w})}{m_\text{k}(T_\text{k} - T_\text{m})}
            \end{equation}
    
    
    \section{Diskussion}

=======
          
          \subsection{Einschränkung des Dulong-Petitschen-Gesetz}
            Während der über das Dulong-Petit-Gesetz gegebene Wert von 3R bei hohen Temperatur von ca. 20°C und höher gut mit gemessenen Werten übereinstimmt,
            fallen die Messwerte für niedrige Temperaturen stark ab. Dies liegt an der Verquantelung der harmonischen Schwingungen in der Gitterstruktur. 
            Durch diese können die einzelnen Energien nur Vielfache des Produkts der Schwingfrequenz und des Planckschen Wirkungsquantum annehmen. Wenn 
            die Schwingungen bei kleinen Temperaturen immer kleiner werden, kann es passieren, dass die Schwingungen so gering werden, dass ihre Energie
            kleiner als das $1 \cdot \hbar \omega$ und sie liefern garkeinen Beitrag mehr, sodass die Molwärme gegen 0 geht. Dieses Verhalten kann man aus Formel
            xxx entnehmen, die dem gerade beschriebenen quantenmechanischen Ansatz entspricht und die innere Energie eines schwingenden Teilchens beschreibt.
            \begin{equation}
              U_{\text{Teilchen}} = \frac{3 N_{\text{A}} \hbar \omega}{e^{\frac{\hbar \omega}{k_bT}}-1}
              \label{eqn:QU}
            \end{equation} 
            Der Exponentialterm geht für $T \rightarrow 0$ gegen $\infty$ und die Molwärme dementsprechend gegen 0. Um die Molwärme für Große Temperaturen
            zu betrachten wird die Taylor-Entwicklung des Exponentialterms (Formel \eqref{eqn:TEU}) genutzt. Mit dieser geht die Molwärme für hohe Temperaturen gegen 
            den klassischen Wert von $3R$.
            \begin{equation}
              U_{\text{Teilchen}} \approx \frac{3 N_{\text{A}} \hbar \omega}{1 + \frac{\hbar \omega}{k_{\text{b}T}}}
              \label{eqn:TEU}
            \end{equation} 

        
        
>>>>>>> d35bdb2cf9ba1e761171438266a2eef3311a4207
\end{document}