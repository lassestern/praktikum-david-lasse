\documentclass[titlepage = firstcover]{scrartcl}
\usepackage[aux]{rerunfilecheck}
\usepackage{fontspec}
\usepackage[main=ngerman, english, french]{babel}

% mehr Pakete hier
\usepackage{expl3}
\usepackage{xparse}

%Mathematik------------------------------------------------------
\usepackage{amsmath}   % unverzichtbare Mathe-Befehle
\usepackage{amssymb}   % viele Mathe-Symbole
\usepackage{mathtools} % Erweiterungen für amsmath
\usepackage[
  math-style=ISO,    % \
  bold-style=ISO,    % |
  sans-style=italic, % | ISO-Standard folgen
  nabla=upright,     % |
  partial=upright,   % /
]{unicode-math}% "Does exactly what it says on the tin."

% Laden von OTF-Mathefonts
% Ermöglich Unicode Eingabe von Zeichen: α statt \alpha

\setmathfont{Latin Modern Math}
%\setmathfont{Tex Gyre Pagella Math} % alternativ zu Latin Modern Math
\setmathfont{XITS Math}[range={scr, bfscr}]
\setmathfont{XITS Math}[range={cal, bfcal}, StylisticSet=1]

\AtBeginDocument{ % wird bei \begin{document}
  % werden sonst wieder von unicode-math überschrieben
  \RenewDocumentCommand \Re {} {\operatorname{Re}}
  \RenewDocumentCommand \Im {} {\operatorname{Im}}
}
\usepackage{mleftright}
\setlength{\delimitershortfall}{-1sp}

%Sprache----------------------------------------------------------
\usepackage{microtype}
\usepackage{xfrac}
\usepackage[autostyle]{csquotes}    % babel
\usepackage[unicode, pdfusetitle]{hyperref}
\usepackage{bookmark}
\usepackage[shortcuts]{extdash}
%Einstellungen hier, z.B. Fonts
\usepackage{booktabs} % Tabellen

%Defininierte funktionen
\DeclareMathOperator{\f}{xyz}

\ExplSyntaxOn % bequeme Syntax für Definition von Befehlen

\NewDocumentCommand \I {} {         %Befehl \I definieren,keine Argumente
  \symup{i}                         %Ergebnis von \I
} 
\NewDocumentCommand \dif {m} % m = mandatory (Pflichtargument für \dif)
{
  \mathinner{\symup{d} #1}
}

\ExplSyntaxOff % Syntax wieder ausschalten. Wichtig!


\title{Das Dulong-Petitsche-Gesetz}
\author{
  David Gutnikov\\
  \href{mailto:david.gutnikov@udo.edu}{david.gutnikov@udo.edu}
 \and 
  Lasse Sternemann\\
  \href{mailto:lasse.sternemann@udo.edu}{lasse.sternemann@udo.edu}
}
\date{Durchführung am 03.12.2019}

\begin{document}
    \maketitle
    \tableofcontents
    \newpage

    \section{Zielsetzung}
        sss


    \section{Theoretische Grundlagen}
        \subsection{Wärmekapazität}
            Wenn ein Stoff, erhitzt wird und im Prozess keine andere Energie aufnimmt oder abgibt,
            \begin{equation*}
                \Delta U = \Delta Q + \Delta A = \Delta Q
            \end{equation*}
            nimmt er die Wärmemenge $\Delta Q$ auf. 
            Die aufgenommene Wärmemenge $\Delta Q$ hängt dabei von der Änderung der Temperatur $\Delta T$,
            der Masse des Körpers $m$, sowie der spezifischen Wärmekapazität des Elements $c$ ab. Dieser Zusammenhang definiert die spezifische
            Wärmekapazität eines Stoffes, wie in Formel \eqref{eqn:wk} beschrieben.  
            \begin{equation}
                c = \frac{\Delta Q}{\Delta T \cdot m }
                \label{eqn:wk}
            \end{equation}
        
        \subsection{Molwärme nach dem Dulong-Petitschen-Gesetz}
            Das Dulong-Petitsche-Gesetz geht davon aus, dass alle Festkörper die gleiche spezifische Wärmekapazität inne habe. Dies wird aus der
            Teilchenenergie, der im Körper gebunden Atome hergeleitet, indem davon ausgegangen wird, dass sie sich in einer festen Position
            innerhalb einer Gitterstruktur befinden. Ihre einzig mögliche Bewegung ist eine harmonische Schwingung um ihre Ruheposition. Wie für
            harmonische üblich ist die kinetische Energie der Teilchen gleich der potentiellen Energie, sodass sich für die gesamte innere Energie eines 
            Teilchens im Festkörper pro Freiheitsgrad Formel \eqref{eqn:eteilchen} ergibt, in der $k_{\text{b}}$ für die Boltzmann-Konstante steht.
            \begin{equation}
                U_{\text{Teilchen pro Freiheitsgrad}} = 2 \cdot E_{\text{Kin}} = k_{\text{b}} \cdot T \longrightarrow_{3 Freiheitsgrade} U_{\text{Teilchen}} = 3 \cdot k_{/text{b}} \cdot T
                \label{eqn:eteilchen}
            \end{equation} 
            Da der Festkörper natürlich aus mehreren Teilchen besteht, wird die gesamte innere Energie des Festkörpers durch Formel \eqref{eqn:ekörper} beschrieben, 
            wobei $N_{\text{A}}$ die Avogrado-Konstante ist, die multipliziert mit der Boltzmann-Konstante die allgemeine Gaskonstante $R$ ergibt. 
            \begin{equation}
                U_{\text{Körper}} = 3 \cdot N_{\text{A}} \cdot k_{\text{b}} = 3 \cdot R \cdot T := Q
                \label{eqn:ekörper}
            \end{equation}
            Über diese Energie lässt sich die spezifische Molwärme beschreiben, die beschreibt wie viel Wärmemenge hinzugefügt werden muss, um ein Mol eines
            Elements um eine Temperatur $\Delta T$ zu erhitzen, sofern der Druck konstant ist. Es ergibt sich für alle Festkörper aus einem Element
            \begin{equation}
                c_{\text{v}} = \frac{dU}{dT} = 3R \qquad \text{mit} \qquad dU = dQ + dA = dQ .
            \end{equation}
            
        
        
\end{document}