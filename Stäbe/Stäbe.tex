\documentclass[titlepage = firstcover]{scrartcl}
\usepackage[aux]{rerunfilecheck}
\usepackage{fontspec}
\usepackage[main=ngerman, english, french]{babel}

% mehr Pakete hier
\usepackage{expl3}
\usepackage{xparse}
\usepackage{pdfpages}
\usepackage{csvsimple}
\usepackage{pgfplotstable}

%Mathematik------------------------------------------------------
\usepackage{amsmath}   % unverzichtbare Mathe-Befehle
\usepackage{amssymb}   % viele Mathe-Symbole
\usepackage{mathtools} % Erweiterungen für amsmath
\usepackage[
  math-style=ISO,    % \
  bold-style=ISO,    % |
  sans-style=italic, % | ISO-Standard folgen
  nabla=upright,     % |
  partial=upright,   % /
]{unicode-math}% "Does exactly what it says on the tin."

% Laden von OTF-Mathefonts
% Ermöglich Unicode Eingabe von Zeichen: α statt \alpha

\setmathfont{Latin Modern Math}
%\setmathfont{Tex Gyre Pagella Math} % alternativ zu Latin Modern Math
\setmathfont{XITS Math}[range={scr, bfscr}]
\setmathfont{XITS Math}[range={cal, bfcal}, StylisticSet=1]

\AtBeginDocument{ % wird bei \begin{document}
  % werden sonst wieder von unicode-math überschrieben
  \RenewDocumentCommand \Re {} {\operatorname{Re}}
  \RenewDocumentCommand \Im {} {\operatorname{Im}}
}
\usepackage{mleftright}
\setlength{\delimitershortfall}{-1sp}

%Sprache----------------------------------------------------------
\usepackage{microtype}
\usepackage{xfrac}
\usepackage[autostyle]{csquotes}    % babel
\usepackage[unicode, pdfusetitle]{hyperref}
\usepackage{bookmark}
\usepackage[shortcuts]{extdash}
%Einstellungen hier, z.B. Fonts
\usepackage{booktabs} % Tabellen
\usepackage{a4}
\usepackage{float}
%Subfiguren
\usepackage{graphicx}
\usepackage{grffile}
\usepackage{subcaption}

\setlength{\parindent}{0pt}


\title{Biegung elastischer Stäbe}
\author{
  David Gutnikov \\
  \href{mailto:david.gutnikov@edu.udo}{david.gutnikov@edu.udo} \and
  Lasse Sternemann \\
  \href{mailto:lasse.sternemann@edu.udo}{lasse.sternemann@edu.udo}}
\date{Durchführung am 26.11.19}

\begin{document}
    \maketitle
    \tableofcontents
    \newpage
    
    \section{Zielsetzung}
      Es sollen für verschiedene Stäbe die Proportionalitätskoeffizienten (Elastizitätsmodul) zwischen ausreichend kleinen Auslenkungen des Stabes und der auf ihn ausgeübten Spannung bestimmt werden.

    \section{Theorie}
      \subsection{Elastizitätsmodul}
        Wenn Kräfte auf einen elastischen Körper wirken, verformt sich dieser, wie in Abbildung \ref{fig:verformung}
        zu sehen ist. Meistens werden diese Kräfte pro Fläche angegeben, was als Spannung bezeichnet wird.
        Diese Spannungen haben einen linearen Zusammenhang zu den durch sie verursachten, relativen Auslenkungen am
        Körper (Auslenkung relativ zur Länge des Körpers), sofern sie ausreichend klein sind.
        Der Proportionalitätskoeffizient dieser Abhängigkeit ist der Elastizitätsmodul. Jedes Material besitzt ein
        eigenes Elastizitätsmodul, welches als Materialkonstante in verschiedenen Fachbereichen Anwendung findet.
        \begin{figure}[h]
          \centering
          \includegraphics[width=0.75\linewidth]{Verformung.png}
          \caption{Der Körper wird durch $\sigma$ verformt und erfährt eine Ausdehnung von $\Delta L$.}
          \label{fig:verformung}
        \end{figure}
        
        Die Formel bzw. das \glqq Hook'sche Gesetz\grqq{} sieht wie folgt aus mit der Auslenkung $\Delta L$ und der Länge des Körpers $L$:
        \begin{equation*}
          \sigma = E \cdot \frac{\Delta L}{L}
        \end{equation*}
        
        \subsection{Biegung eines einseitig eingespannten Stabes}
          Wie in Abbildung \ref{fig:theorieEins} zu sehen ist, wird der Stab durch die senkrecht zu ihm angreifende Kraft $F$ um $D$ 
          ausgelenkt. Die Auslenkung kommt zustande, indem der Stab sich durch das äußere Drehmoment biegt, bis die 
          inneren Normalspannungen ein Kräftegleichgewicht erzeugen und der Stab in der zugehörigen Auslenkung verbleibt.
          Diese Auslenkung hängt vom Abstand zum Befestigungspunkt $x$, dem Elastizitätsmodul des Körpermaterials $E$, 
          dem Flächenträgheitsmoment $I$ und der Länge des Körper s$L$, sowie der angreifenden Kraft $F$ ab. Dieser 
          Zusammenhang ist in Formel \eqref{eqn:formelEins} dargestellt:
          \begin{equation}
            D(x) = \frac{F}{2EI} \cdot \Bigl(Lx^2 - \frac{x^3}{3}\Bigr)
            \label{eqn:formelEins}
          \end{equation}

          \begin{figure}[h]
            \centering
            \includegraphics[width=0.75\linewidth]{Einseitig_Theorie.png}
            \caption{Der Stab wird einseitig eingespannt und an einer Seite ein Gewicht angehängt.}
            \label{fig:theorieEins}
          \end{figure}
      
        \subsection{Biegung eines zweiseitig eingespannten Stabes}
          Wie in \ref{eqn:formelZweis} zu sehen ist, wird in der Mitte zwischen den Auflagepunkten eine senkrechte Kraft an
          den Stab angelegt. Die anliegende Kraft wird auf die beiden Auflagepunkte übertragen und erzeugt auch hier nach 
          dem obrigen Prinzip eine Auslenkung. Diese ist von den selben Größen wie Formel \eqref{eqn:formelEins} abhängig, 
          unterscheidet sich jedoch um den Faktor 1/24. Demnach ergibt sich für die zweiseitige Auflage Formel 
          \eqref{eqn:formelZweis}:
          \begin{equation}
            D(x) = \frac{F}{48EI} \cdot \Bigl(3L^2x - 4x^3\Bigr)
            \label{eqn:formelZweis}
          \end{equation}

          \begin{figure}[h]
            \centering
            \includegraphics[width=0.5\linewidth]{Zweiseitig_Theorie.png}
            \caption{Der Stab wird einseitig eingespannt und an einer Seite ein Gewicht angehängt.}
            \label{fig:theorieEins}
          \end{figure}
      

    \section{Versuchsdurchführung}
      Zur Bestimmung des Elastizitätsmoduls der Stabmaterialien werden Gewichte an dem Stab angebracht und daraufhin dessen Auslenkung abhängig von der Entfernung
      zum Angriffspunkt des Gewichts gemessen. Es werden zwei Stäbe verwendet, von denen einer einen runden Querschnitt und der andere einen rechteckigen hat. Beide
      Stäbe werden in zwei Positionen eingespannt und gemessen. Zum einen werden sie nur an einem Auflagepunkt eingespannt und das Gewicht möglichst weit davon entfernt
      angebracht. Dieser Aufbau ist in Abbildung \ref{fig:theorieEins} zu sehen. Bei dem zweiten, in Abbildung \ref{fig:formelZweis} zu sehenden Aufbau, liegt der Stab auf zwei Auflagepunkten und das
      Gewicht greift in der Mitte der beiden Auflagepunkte an.\newline

      Zur Messung der Auslenkung werden Messuhren wie in Abbildung \ref{fig:fotoUhr} verwendet. Diese bestehen aus einem, an einer Feder befestigten, Messtaster und einer runden Skala. Wenn der 
      Messtaster eingedrückt wird, wird die Distanz auf der Rundskala, die von 0,01 bis 10 Millimeter geht, angezeigt. Mit diesen Messuhren wird vor Beginn der
      Messung der Auslenkung des Stabes über den unausgelenkten Stab gefahren, um die Auslenkung in der Ruhelage zu messen. Dies ist notwendig, da die Stäbe 
      bereits sehr of ausgelenkt worden sind und nie wieder in die ehemalige gerade Form zurückgekehrt sind. Nachdem auf diese Weise, die Ruheauslenkung 
      festgestellt worden ist, wird das Gewicht angehangen und erneut mit den Messuhren über den Stab gefahren. Dabei werden in Abständen von 3 cm die realtiven 
      Auslenkungen gemessen. Im Nachhinein werden die Ruheauslenkungen von den realtiven Auslenkungen abgezogen, um die tatsächliche Auslenkung des Stabes zu
      messen. Da die Messuhr bei zweiseitiger Einspannung in der Mitte blockiert wird, muss in diesem Fall mit zwei verschiedenen Messuhren gemessen werden. Dies
      kann zu unterschieden in den Messergebnissen führen, da bei unserem Experiment besonders eine der Uhren sehr unzuverlässige Werte lieferte.
      
      \begin{figure}[h]
        \centering
        \includegraphics[width=0.25\linewidth]{Messuhr.jpg}
        \caption{Zu sehen ist eine der mechanischen Messuhren, die zur Bestimmung der Auslenkung genutzt worden ist. Sie ist auf 0,01mm genau und kann Längen bis zu 10mm messen.}
        \label{fig:fotoUhr}
      \end{figure}   
    
    \newpage
    \section{Auswertung}
      Für das Experiment wurde ein Stab mit rechteckiger Querschnittfläche der Länge $l_e = $
      \subsection{Eckiger Stab}
        \begin{table}[h]
          \centering
          \caption{Messwerte des eckigen, einseitig eingespannten Stabes.
                   $x$ ist der Abstand vom Einspannpunkt,
                   $D_0(x)$ ist die Auslenkung ohne Krafteinwirkung,
                   $D_a(x)$ ist die Auslenkung mit Krafteinwirkung,
                   $D(x)$ ist die tatsächliche Auslenkung, also die Differenz von $D_a$ und $D_0$.}
          \label{tab:tabEeins}
          \begin{tabular}{c c c c}
            \toprule
            {$x$ [m]} & {$D_0(x)$ [$10^{-3}m$]} & {$D_a(x)$ [$10^{-3}m$]} & {$D(x)$ [$10^{-3}m$]}\\
            \midrule
            0.03 & 0.07 & 0.17 & 0.10\\
            0.06 & 0.09 & 0.39 & 0.30\\
            0.09 & 0.11 & 0.57 & 0.46\\
            0.12 & 0.15 & 0.86 & 0.75\\
            0.15 & 0.17 & 1.20 & 1.03\\
            0.18 & 0.22 & 1.63 & 1.47\\
            0.21 & 0.24 & 1.97 & 1.73\\
            0.24 & 0.10 & 2.40 & 2.30\\
            0.27 & 0.09 & 2.80 & 2.71\\
            0.30 &-0.05 & 3.26 & 3.31\\
            0.33 &-0.11 & 3.74 & 3.85\\
            0.36 &-0.21 & 4.23 & 4.44\\
            0.39 &-0.25 & 4.77 & 5.02\\
            0.42 &-0.31 & 5.33 & 5.64\\
            0.45 &-0.33 & 5.89 & 6.22\\
            \bottomrule            
          \end{tabular}
        \end{table}

        \begin{figure}
          \centering
          \includegraphics[width=0.7\linewidth]{eeins.pdf}
          \caption{In der Grafik sind die Auslenkungen D gegen Lx²-x³/3 aufgetragen. Durch diese Wertepaare wurde zudem per linearer Regression eine Ausgleichsgerade gelegt. Deren Steigung beträgt 0,08901088 und deren y-Achsenabschnitt 0,00014607.}
          \label{fig:graphEeins}
        \end{figure}

        \begin{table}[h]
          \centering
          \caption{Messwerte des eckigen, zweiseitig eingespannten Stabes.
                   $x$ ist der Abstand vom rechten Einspannpunkt,
                   $D_0(x)$ ist die Auslenkung ohne Krafteinwirkung,
                   $D_a(x)$ ist die Auslenkung mit Krafteinwirkung,
                   $D(x)$ ist die tatsächliche Auslenkung, also die Differenz von $D_a$ und $D_0$,
                   $x_{rel}$ ist der Abstand zum jeweils näheren Einspannpunkt.}
          \label{tab:tabEzwei}
          \begin{tabular}{c c c c c}
            \toprule
            {$x$ [m]} & {$D_0(x)$ [$10^{-3}m$]} & {$D_a(x)$ [$10^{-3}m$]} & {$D(x)$ [$10^{-3}m$]} & {$x_{rel}$ [m]}\\
            \midrule
            0.03 &  0.05 &  0.17 & 0.12 & 0.03\\
            0.06 &  0.10 &  0.31 & 0.21 & 0.06\\
            0.09 &  0.16 &  0.46 & 0.30 & 0.09\\
            0.12 &  0.25 &  0.63 & 0.38 & 0.12\\
            0.15 &  0.29 &  0.63 & 0.38 & 0.15\\
            0.18 &  0.40 &  0.94 & 0.54 & 0.18\\
            0.21 &  0.38 &  0.96 & 0.56 & 0.21\\
            0.24 &  0.41 &  1.02 & 0.61 & 0.24\\
            0.30 & -0.04 &  0.74 & 0.78 & 0.24\\
            0.33 & -0.15 &  0.62 & 0.77 & 0.21\\
            0.36 & -0.26 &  0.45 & 0.71 & 0.18\\
            0.39 & -0.36 &  0.28 & 0.64 & 0.15\\
            0.42 & -0.45 &  0.10 & 0.55 & 0.12\\
            0.45 & -0.56 & -0.12 & 0.44 & 0.09\\
            0.48 & -0.65 & -0.33 & 0.32 & 0.06\\
            0.51 & -0.77 & -0.55 & 0.22 & 0.03\\
            0.54 & -0.92 & -0.70 & 0.22 & 0.01\\
            \bottomrule            
          \end{tabular}
        \end{table}
  
        \begin{figure}[h]
          \centering
          \includegraphics[width=0.7\linewidth]{ezwei.pdf}
          \caption{In der Grafik sind die Auslenkungen D gegen Lx²-x³/3 aufgetragen. Durch diese Wertepaare wurde zudem per linearer Regression zwei Ausgleichsgerade gelegt. Es wurden zwei Ausgleichsgeraden generiert, da die zwei Uhren stark unterschiedlich gemessen haben. Die Gerade von Uhr 1 hat die Steigung $3,53205921\cdot10^{-3}$ und den y-Achsenabschnitt $1,65755043\cdot10^{-5}$, die von Uhr 2 die Steigung $4,03285\cdot10^{-3}$ und den y-Achsenabschnitt $0.13614\cdot10^{-3}$.}
          \label{fig:graphEzwei}
        \end{figure}
%%%%%%%%%%%%%%%%%%%%%%%%%%%%%%%%%%%%%%%%%%%%%%%%%%%%%%%%%%%%%%%%%%%%%%%%%%%%%%%%%%%%%%%%%%%%%%%%%%%%%%%%%%%  
      \newpage
      \subsection{Runder Stab}
      \begin{table}[h]
        \centering
        \caption{Messwerte des runden, einseitig eingespannten Stabes wie in Tabelle \ref{tab:tabEeins}.}
        \label{tab:tabReins}
        \begin{tabular}{c c c c}
          \toprule
          {$x$ [m]} & {$D_0(x)$ [$10^{-3}m$]} & {$D_a(x)$ [$10^{-3}m$]} & {$D(x)$ [$10^{-3}m$]}\\
          \midrule
          0.03 &-0.01 & 0.08 & 0.09\\
          0.06 &-0.14 & 0.06 & 0.20\\
          0.09 &-0.28 & 0.10 & 0.38\\
          0.12 &-0.39 & 0.17 & 0.56\\
          0.15 &-0.52 & 0.29 & 0.81\\
          0.18 &-0.60 & 0.49 & 1.09\\
          0.21 &-0.76 & 0.63 & 1.39\\
          0.24 &-0.84 & 0.89 & 1.73\\
          0.27 &-0.93 & 1.15 & 2.08\\
          0.30 &-1.05 & 1.43 & 2.48\\
          0.33 &-1.13 & 1.73 & 2.86\\
          0.36 &-1.22 & 2.06 & 3.28\\
          0.39 &-1.31 & 2.42 & 3.71\\
          0.42 &-1.34 & 2.81 & 4.15\\
          0.45 &-1.37 & 3.21 & 4.58\\
          \bottomrule            
        \end{tabular}
      \end{table}
    
      \begin{figure}
        \centering
        \includegraphics[width=0.7\linewidth]{reins.pdf}
        \caption{In der Grafik sind die Auslenkungen D gegen Lx²-x³/3 aufgetragen. Durch diese Wertepaare wurde zudem per linearer Regression eine Ausgleichsgerade gelegt. Deren Steigung beträgt 0,06525698 und deren y-Achsenabschnitt 0,00014415.}
        \label{fig:graphReins}
      \end{figure}

      \begin{table}[h]
        \centering
        \caption{Messwerte des runden, zweiseitig eingespannten Stabes wie in Tabelle \ref{tab:tabEzwei}.}
        \label{tab:tabRzwei}
        \begin{tabular}{c c c c c}
          \toprule
          {$x$ [m]} & {$D_0(x)$ [$10^{-3}m$]} & {$D_a(x)$ [$10^{-3}m$]} & {$D(x)$ [$10^{-3}m$]} & {$x_{rel}$ [m]}\\
          \midrule
          0.03 & 0.28 & 0.40 & 0.12 & 0.03\\
          0.06 & 0.41 & 0.65 & 0.24 & 0.06\\
          0.09 & 0.42 & 0.85 & 0.43 & 0.09\\
          0.12 & 0.41 & 0.97 & 0.56 & 0.12\\
          0.15 & 0.37 & 1.05 & 0.68 & 0.15\\
          0.18 & 0.26 & 1.12 & 0.86 & 0.18\\
          0.21 & 0.24 & 1.08 & 0.84 & 0.21\\
          0.24 & 0.19 & 1.07 & 0.88 & 0.24\\
          0.30 & 0.70 & 1.81 & 1.11 & 0.25\\
          0.33 & 0.61 & 1.68 & 1.07 & 0.22\\
          0.36 & 0.55 & 1.52 & 0.97 & 0.19\\
          0.39 & 0.49 & 1.34 & 0.85 & 0.16\\
          0.42 & 0.42 & 1.16 & 0.74 & 0.13\\
          0.45 & 0.34 & 0.96 & 0.62 & 0.10\\
          0.48 & 0.25 & 0.75 & 0.50 & 0.07\\
          0.51 & 0.16 & 0.50 & 0.34 & 0.04\\
          0.54 & 0.04 & 0.29 & 0.25 & 0.01\\
          \bottomrule            
        \end{tabular}
      \end{table}

      \begin{figure}[h]
        \centering
        \includegraphics[width=0.7\linewidth]{rzwei.pdf}
        \caption{In der Grafik sind die Auslenkungen D gegen Lx²-x³/3 aufgetragen. Durch diese Wertepaare wurde zudem per linearer Regression zwei Ausgleichsgerade gelegt. Es wurden zwei Ausgleichsgeraden generiert, da die zwei Uhren stark unterschiedlich gemessen haben. Die Gerade von Uhr 1 hat die Steigung $5,60760\cdot10^{-3}$ und den y-Achsenabschnitt $-5,94841\cdot10^{-6}$, die von Uhr 2 die Steigung $6,61459\cdot10^{-3}$ und den y-Achsenabschnitt $2,98337\cdot10^{-5}$.}
        \label{fig:graphRzwei}
      \end{figure}

\end{document}