\documentclass{scrartcl}
\usepackage[aux]{rerunfilecheck}

\usepackage{fontspec}

\usepackage[main=ngerman]{babel}

\usepackage{amsmath}
\usepackage{amssymb}
\usepackage{mathtools}

\usepackage[
  math-style=ISO,
  bold-style=ISO,
  sans-style=italic,
  nabla=upright,
  partial=upright,
]{unicode-math}

\usepackage[unicode]{hyperref}
\usepackage{bookmark}
\usepackage{booktabs}

\title{Das Hooksche Gesetz}
\author{Paul Horn \and Lasse Sternemann}
\date{22.10.19}

\begin{document}
    \maketitle

    \section{Versuchsbeschreibung}
        In diesem Versuch geht es darum, den Zusammenhang zwischen der auf die Feder wirkenden Kraft und der Federauslenkung zu bestimmen.
        Die Feder hängt an einem Kraftmesser, der die an der Feder anliegende Kraft F misst. Zudem ist an der Feder ein Seil befestigt, 
        welches über eine Umlenkrolle entlang eines Maßstabs gezogen werden kann. So lässt sich die Auslenkung der Feder in Abhängigkeit zu
        der angreifenden Kraft ablesen.

    \section{Versuchsdurchführung}
        Zuerst wird das Seil so gestrafft, dass sich die Feder trotzdem noch in ihrer Ruhelage befindet. Nun wird graduell die Auslenkung der 
        Feder in Intervallen von je 5 Zentimetern erhöht. Dabei wird nach jeder Erhöhung die auf die Feder wirkende Kraft notiert.

        
    \section{Bestimmung der Federkonstante}
        Nun wollen wir die Federkonstante bestimmen. Dazu stellen wir das Hooksche Gesetzt nach D um.

    \begin{equation}
        F = D \cdot \Delta x  \qquad \Leftrightarrow  \qquad D = \frac{F}{\Delta x}
    \end{equation} 

        \begin{table}
            \centering
            \caption{Messdaten}
            \label{tab:Tabelle_1}
            
            \begin{tabular}{c c}
                \toprule
                $\Delta x$ \ [cm] & $F$ \ [N] & $D$ \\
                \midrule
                5 & 0,15 \\
                10 & 0,29 \\
                15 & 0,44 \\
                20 & 0,59 \\
                25 & 0,74 \\
                30 & 0,89 \\
                35 & 1,04 \\
                40 & 1,19 \\
                45 & 1.34, \\
                50 & 1,49 \\
                \bottomrule
            \end{tabular}    
        \end{table}
        

        \end{document}