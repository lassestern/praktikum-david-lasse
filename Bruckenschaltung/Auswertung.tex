\documentclass[titlepage = firstcover]{scrartcl}
\usepackage[aux]{rerunfilecheck}
\usepackage{fontspec}
\usepackage[main=ngerman, english, french]{babel}

% mehr Pakete hier
\usepackage{expl3}
\usepackage{xparse}

%Mathematik------------------------------------------------------
\usepackage{amsmath}   % unverzichtbare Mathe-Befehle
\usepackage{amssymb}   % viele Mathe-Symbole
\usepackage{mathtools} % Erweiterungen für amsmath
\usepackage[
  math-style=ISO,    % \
  bold-style=ISO,    % |
  sans-style=italic, % | ISO-Standard folgen
  nabla=upright,     % |
  partial=upright,   % /
]{unicode-math}% "Does exactly what it says on the tin."
\usepackage[section, below]{placeins}

% Laden von OTF-Mathefonts
% Ermöglich Unicode Eingabe von Zeichen: α statt \alpha

\setmathfont{Latin Modern Math}
%\setmathfont{Tex Gyre Pagella Math} % alternativ zu Latin Modern Math
\setmathfont{XITS Math}[range={scr, bfscr}]
\setmathfont{XITS Math}[range={cal, bfcal}, StylisticSet=1]

\AtBeginDocument{ % wird bei \begin{document}
  % werden sonst wieder von unicode-math überschrieben
  \RenewDocumentCommand \Re {} {\operatorname{Re}}
  \RenewDocumentCommand \Im {} {\operatorname{Im}}
}
\usepackage{mleftright}
\setlength{\delimitershortfall}{-1sp}

%Sprache----------------------------------------------------------
\usepackage{microtype}
\usepackage{xfrac}
\usepackage[autostyle]{csquotes}    % babel
\usepackage[unicode, pdfusetitle]{hyperref}
\usepackage{bookmark}
\usepackage[shortcuts]{extdash}
%Einstellungen hier, z.B. Fonts
\usepackage{booktabs} % Tabellen


\title{Wärmeleitfähigkeit}
\author{
  David Gutnikov\\
  \href{mailto:david.gutnikov@udo.edu}{david.gutnikov@udo.edu}
 \and 
  Lasse Sternemann\\
  \href{mailto:lasse.sternemann@udo.edu}{lasse.sternemann@udo.edu}
}
\date{Durchführung am 10.12.2019}


\begin{document}
  \maketitle
  \newpage
  \tableofcontents
  \newpage


  \section{Auswertung}

  \subsection{Anmerkungen zu Rechenvorgängen}
  Die in diesen Messungen entstandenen Fehler, wurden bereits als relative Fehler angegeben. Dadurch lassen sich die fortgepflanzten Fehler wie folgt 
  beschreiben:
  \begin{equation}
      \sigma = \sqrt{\sum_{i=1}^N \sigma_i^2}
      \label{eqn:Fehler}
  \end{equation}
  Zudem werden Mittelwerte über den Mittelwertssatz berrechnet:
  \begin{equation}
    \overline{x} = \frac{1}{N} \cdot \sum_{i=1}^{N} x_i
    \label{eqn:Mittelwertssatz}
  \end{equation}

  \subsection{Wheatonsche Brücke}
  Mit den in Tabelle \ref{tab:Wheaton} gegebenen Werten lassen sich über Formel xx mehrere Werte für die beiden Widerstände berechnen. Aus diesen wird über 
  Formel \ref{eqn:Mittelwertssatz} ein Mittelwert berechnet. Der zugehörige Fehler ergibt sich aus Formel \ref{eqn:Fehler} und den folgenden prozentualen
  Fehlern:
  \begin{align*}
      \sigma_{\text{R2}} = 0,2 \% \\
      \sigma_{\text{R3/R4}} = 0,5 \% \\
  \end{align*}
  So ergeben sich folgende Werte für die Widerstände:
  \begin{align*}
      R_{14} = (894 \pm 7) \Omega
      R_{13} = (314 \pm 3) \Omega
  \end{align*}

  \begin{table}[h]
    \centering
    \caption{In der Tabelle sind die für die Rechnung notwendigen Widerstände für beide Widerstände angegeben.}
    \label{tab:Wheaton}
    \begin{tabular}{c c c c}
      \toprule
      {Wert} & {$R_{\text{2}} [\Omega]$} & {$R_{\text{3}} [\Omega]$} & {$R_{\text{4}} [\Omega]$} \\
      \midrule 
      14 & 1000 & 472 & 528  \\
       & 664 & 573 & 427 \\
       & 332 & 730 & 270 \\
      13 & 1000 & 236 & 764 \\
       & 664 & 323 & 677 \\
       & 332 & 489 & 511 \\
      \bottomrule
    \end{tabular}
  \end{table}

  \subsection{Kapazitätsbrücke}
  Nun werden Formel xx mit den Werten aus Tabelle \ref{tab:Kap} und wieder die angegeben relativen Fehler genutzt, um die Kapazität zu bestimmen.
  \begin{align*}
    \sigma_{\text{R2}} = 3 \% \\
    \sigma_{\text{R3/R4}} = 0,5 \% \\
    \sigma_{\text{C2}} = 0,2 \% \\
    R_9 = (463 \pm 15) \Omega\\
    C_9 = (435 \pm 4) \cdot 10^{-9} F
  \end{align*}

  \begin{table}[h]
    \centering
    \caption{In der Tabelle sind die für die Rechnung notwendigen Widerstände für beide Widerstände angegeben.}
    \label{tab:Kap}
    \begin{tabular}{c c c c c}
      \toprule
      {Wert} & {$C_{\text{2}} [nF]$} & {$R_{\text{2}} [\Omega]$} & {$R_{\text{3}} [\Omega]$} & {$R_{\text{4}} [\Omega]$} \\
      \midrule 
      9 & 399 & 500 & 480 & 520  \\
       & 597 & 339 & 580 & 420 \\
       & 992 & 204 & 693 & 307 \\
      \bottomrule
    \end{tabular}
  \end{table}

  \subsection{Indu}
  Auch bei der Induktivitätsbrücke wird die Formel der Wheaton-Brücke wieder verwendet, um den Widerstand zu bestimmen. Die Induktivität der Spule 16 wird
  über Formel xx berechnet. Der zugehörige Fehler folgt dabei aus den angegebenen relativen Fehlern.
  \begin{align*}
    \sigma_{\text{R2}} = 3 \% \\
    \sigma_{\text{R3/R4}} = 0,5 \% \\
    \sigma_{\text{L2}} = 0,2 \% \\
    R_{16} = (405 \pm 25) \Omega\\
    L_{16} = (58 \pm 16) \cdot 10^{-3} H
  \end{align*}

  \begin{table}[h]
    \centering
    \caption{In der Tabelle sind die für die Rechnung notwendigen Widerstände für beide Widerstände angegeben.}
    \label{tab:Indu}
    \begin{tabular}{c c c c c}
      \toprule
      {Wert} & {$L_{\text{2}} [mH]$} & {$R_{\text{2}} [\Omega]$} & {$R_{\text{3}} [\Omega]$} & {$R_{\text{4}} [\Omega]$} \\
      \midrule 
      16 & 27,5 & 122 & 757 & 243  \\
       & 24,6 & 115 & 789 & 211 \\
      \bottomrule
    \end{tabular}
  \end{table}

  \subsection{Maxwellbrücke}
  Zur Bestimmung des Innenwiderstands wird wieder die Formel der Wheaton-Brücke genutzt und auch der Fehler ergibt sich wie zuvor. Die Induktivität der Spule
  wird über Formel xx berechnet.
  \begin{align*}
    \sigma_{\text{R2}} = 3 \% \\
    \sigma_{\text{R3/R4}} = 3 \% \\
    \sigma_{\text{L2}} = 0,2 \% \\
    R_{16,2} = (416 \pm 25) \Omega\\
    L_{16,2} = (93 \pm 10) \cdot 10^{-3} H
  \end{align*}

  \begin{table}[h]
    \centering
    \caption{In der Tabelle sind die für die Rechnung notwendigen Widerstände für beide Widerstände angegeben.}
    \label{tab:Maxwell}
    \begin{tabular}{c c c c c}
      \toprule
      {Wert} & {$C_{\text{4}} [nF]$} & {$R_{\text{2}} [\Omega]$} & {$R_{\text{3}} [\Omega]$} & {$R_{\text{4}} [\Omega]$} \\
      \midrule 
      16 & 399 & 1000 & 293 & 707  \\
       & 399 & 332 & 557 & 443 \\
       & 399 & 664 & 386 & 614 \\
       & 450 & 332 & 555 & 445 \\
      \bottomrule
    \end{tabular}
  \end{table}

  \subsection{Wien-Robinson-Brücke}

  \begin{table}[h]
    \centering
    \caption{In der Tabelle sind die Temperaturen der einzelnen Messplatten zum Zeitpunkt t=700S angegeben.}
    \label{tab:Wien}
    \begin{tabular}{c c }
      \toprule
      {\text{Frequenz} [s]} & {U [\text{mV}]} \\
      \midrule 
        41 & 440    \\
        91  & 440      \\
        141 &  320 \\
        191 &  128 \\
        201 &  102 \\
        211 &  76 \\
        221 &  52,8 \\
        231 &  28,8 \\
        241 &  4 \\
        251 &  25,6 \\
        261 &  46,4 \\
        271 &  65,6 \\
        281 &  84,8 \\
        291 &  102 \\
        341 &  200 \\
        391 &  280 \\
        441 &  360 \\
        491 &  400 \\
        541 &  440 \\
        591 &  520 \\
      \bottomrule
    \end{tabular}
  \end{table}

\end{document}