\documentclass[titlepage = firstcover]{scrartcl}
\usepackage[aux]{rerunfilecheck}
\usepackage{fontspec}
\usepackage[main=ngerman, english, french]{babel}

% mehr Pakete hier
\usepackage{expl3}
\usepackage{xparse}

%Mathematik------------------------------------------------------
\usepackage{amsmath}   % unverzichtbare Mathe-Befehle
\usepackage{amssymb}   % viele Mathe-Symbole
\usepackage{mathtools} % Erweiterungen für amsmath
\usepackage[
  math-style=ISO,    % \
  bold-style=ISO,    % |
  sans-style=italic, % | ISO-Standard folgen
  nabla=upright,     % |
  partial=upright,   % /
]{unicode-math}% "Does exactly what it says on the tin."
\usepackage[section, below]{placeins}

% Laden von OTF-Mathefonts
% Ermöglich Unicode Eingabe von Zeichen: α statt \alpha

\setmathfont{Latin Modern Math}
%\setmathfont{Tex Gyre Pagella Math} % alternativ zu Latin Modern Math
\setmathfont{XITS Math}[range={scr, bfscr}]
\setmathfont{XITS Math}[range={cal, bfcal}, StylisticSet=1]

\AtBeginDocument{ % wird bei \begin{document}
  % werden sonst wieder von unicode-math überschrieben
  \RenewDocumentCommand \Re {} {\operatorname{Re}}
  \RenewDocumentCommand \Im {} {\operatorname{Im}}
}
\usepackage{mleftright}
\setlength{\delimitershortfall}{-1sp}

%Sprache----------------------------------------------------------
\usepackage{microtype}
\usepackage{xfrac}
\usepackage[autostyle]{csquotes}    % babel
\usepackage[unicode, pdfusetitle]{hyperref}
\usepackage{bookmark}
\usepackage[shortcuts]{extdash}
%Einstellungen hier, z.B. Fonts
\usepackage{booktabs} % Tabellen


\title{Wärmeleitfähigkeit}
\author{
  David Gutnikov\\
  \href{mailto:david.gutnikov@udo.edu}{david.gutnikov@udo.edu}
 \and 
  Lasse Sternemann\\
  \href{mailto:lasse.sternemann@udo.edu}{lasse.sternemann@udo.edu}
}
\date{Durchführung am 10.12.2019}


\begin{document}
  \maketitle
  \newpage
  \tableofcontents
  \newpage

  \section{Durchführung}

  \subsection{Allgemeines Vorgehen}
  Der allgemeine Messvorgang zur Bestimmung der zu messenden Größen beruht darauf, dass in der Schaltung abgesehen von der Eigenschaft, des zu messenden 
  Bauteils, alle anderen Bauteilwerte bekannt sind. Die in der Theorie erläuterten Formeln können dann Anwendung finden, wenn die minimale Brückenspannung
  ermittelt worden ist. Dazu werden Variable Widerstände verstellt. Diese Variablen Widerstände werden durch Potentiometer realisiert, die einen 
  Gesamtwiderstand von $1k\Omega$ inne haben und diesen entweder auf zwei Widerstände distributieren oder als einzelner Widerstand fungieren können.

  \subsection{Wheatonsche Brücke}
  Bei der Wheatonschen Brücke wird nur ein Potentiometer genutzt, um die Widerstände R3 und R4 zu variieren. Dies wird solange durchgeführt bis die 
  Brückenspannung ihr Minimum erreicht. Dieses Minimum wird über ein Oszilloskop, das parallel zur "Brücke" geschaltet ist, ermittelt. Ein Minimum liegt 
  vor, wenn die Amplitude des Wechselstroms bei Widerstandsänderung in beide Richtungen ansteigen würde. Das Schaltbild ist bereits in der Theroie 
  aufgehführt.

  \subsection{Kapazitätsbrücke}
  Bei der Brücke zur Kapazitätsbestimmung werden die drei variablen Widerstände R2, R3 und R4 genutzt.




  \newpage
  %%%%%%%%%%%%%%%%%%%%%%%%%%%%%%%%%%%%%%%%%%%%%%%%%%%%%%%%%%%%%%%%%%%%%%%%%%%%%%%%%%%%%%%%%%%


  \section{Auswertung}

  \subsection{Anmerkungen zu Rechenvorgängen}
  Die in diesen Messungen entstandenen Fehler, wurden bereits als relative Fehler angegeben. Dadurch lassen sich die fortgepflanzten Fehler wie folgt 
  beschreiben:
  \begin{equation}
      \sigma = \sqrt{\sum_{i=1}^N \sigma_i^2}
      \label{eqn:Fehler}
  \end{equation}
  Zudem werden Mittelwerte über den Mittelwertssatz berrechnet:
  \begin{equation}
    \overline{x} = \frac{1}{N} \cdot \sum_{i=1}^{N} x_i
    \label{eqn:Mittelwertssatz}
  \end{equation}
  Aufgrund des Bildens der Mittelwerte lässt sich der Fehler auch über die Standardabweichung des Mittelwerts angeben. Diese wird über folgende Formel 
  berechnet:
  \begin{equation} 
    \Delta f = \sqrt{\sum_{i=1}^N (\frac{df}{dy_i}\cdot \Delta y_i)^2}
    \label{eqn:stanni}
  \end{equation} 
  Als Fehler wird letztendlich der größere der beiden Fehler angegeben.

  \subsection{Wheatonsche Brücke}
  Mit den in Tabelle \ref{tab:Wheaton} gegebenen Werten lassen sich über Formel xx mehrere Werte für die beiden Widerstände berechnen. Aus diesen wird über 
  Formel \ref{eqn:Mittelwertssatz} ein Mittelwert berechnet. Der zugehörige Fehler ergibt sich aus Formel \ref{eqn:Fehler} und den folgenden prozentualen
  Fehlern:
  \begin{align*}
      \sigma_{\text{R2}} = 0,2 \% \\
      \sigma_{\text{R3/R4}} = 0,5 \% \\
  \end{align*}
  So ergeben sich folgende Werte für die Widerstände:
  \begin{align*}
      R_{14} = (894 \pm 7) \Omega \\
      R_{13} = (314 \pm 3) \Omega
  \end{align*}

  \begin{table}[h]
    \centering
    \caption{In der Tabelle sind die für die Rechnung notwendigen Widerstände für beide Widerstände angegeben.}
    \label{tab:Wheaton}
    \begin{tabular}{c c c c c}
      \toprule
      {Wert} & {$R_{\text{x}} [\Omega]$} & {$R_{\text{2}} [\Omega]$} & {$R_{\text{3}} [\Omega]$} & {$R_{\text{4}} [\Omega]$} \\
      \midrule 
      14 & 894 & 1000 & 472 & 528  \\
       & 891 & 664 & 573 & 427 \\
       & 897 & 332 & 730 & 270 \\
      13 & 309 & 1000 & 236 & 764 \\
       & 317 & 664 & 323 & 677 \\
       & 318 & 332 & 489 & 511 \\
      \bottomrule
    \end{tabular}
  \end{table}

  \subsection{Kapazitätsbrücke}
  Nun werden Formel xx mit den Werten aus Tabelle \ref{tab:Kap} und wieder die angegeben relativen Fehler genutzt, um die Kapazität zu bestimmen.
  \begin{align*}
    \sigma_{\text{R2}} = 3 \% \\
    \sigma_{\text{R3/R4}} = 0,5 \% \\
    \sigma_{\text{C2}} = 0,2 \% \\
    R_9 = (463 \pm 15) \Omega\\
    C_9 = (435 \pm 4) \cdot 10^{-9} F
  \end{align*}

  \begin{table}[h]
    \centering
    \caption{In der Tabelle sind die für die Rechnung notwendigen Widerstände für beide Widerstände angegeben.}
    \label{tab:Kap}
    \begin{tabular}{c c c c c c c}
      \toprule
      {Wert} & {$C_{\text{9}} [nF]$} & {$R_{\text{9}} [\Omega]$} & {$C_{\text{2}} [nF]$} & {$R_{\text{2}} [\Omega]$} & {$R_{\text{3}} [\Omega]$} & {$R_{\text{4}} [\Omega]$} \\
      \midrule 
      9 & 432 & 462 &399 & 500 & 480 & 520  \\
       & 432 & 468 & 597 & 339 & 580 & 420 \\
       & 439 & 460 & 992 & 204 & 693 & 307 \\
      \bottomrule
    \end{tabular}
  \end{table}

  \subsection{Induktivitätsbrücke}
  Auch bei der Induktivitätsbrücke wird die Formel der Wheaton-Brücke wieder verwendet, um den Widerstand zu bestimmen. Die Induktivität der Spule 16 wird
  über Formel xx berechnet. Der zugehörige Fehler folgt dabei aus den angegebenen relativen Fehlern.
  \begin{align*}
    \sigma_{\text{R2}} = 3 \% \\
    \sigma_{\text{R3/R4}} = 0,5 \% \\
    \sigma_{\text{L2}} = 0,2 \% \\
    R_{16} = (405 \pm 25) \Omega\\
    L_{16} = (58 \pm 16) \cdot 10^{-3} H
  \end{align*}

  \begin{table}[h]
    \centering
    \caption{In der Tabelle sind die für die Rechnung notwendigen Widerstände für beide Widerstände angegeben.}
    \label{tab:Indu}
    \begin{tabular}{c c c c c c c}
      \toprule
      {Wert} & {$L_{\text{16}} [mH]$} & {$R_{\text{16}} [\Omega]$} & {$L_{\text{2}} [mH]$} & {$R_{\text{2}} [\Omega]$} & {$R_{\text{3}} [\Omega]$} & {$R_{\text{4}} [\Omega]$} \\
      \midrule 
      16 & 86 & 380 & 27,5 & 122 & 757 & 243  \\
       & 55 & 430 & 24,6 & 115 & 789 & 211 \\
      \bottomrule
    \end{tabular}
  \end{table}

  \subsection{Maxwellbrücke}
  Zur Bestimmung des Innenwiderstands wird wieder die Formel der Wheaton-Brücke genutzt und auch der Fehler ergibt sich wie zuvor. Die Induktivität der Spule
  wird über Formel xx berechnet.
  \begin{align*}
    \sigma_{\text{R2}} = 3 \% \\
    \sigma_{\text{R3/R4}} = 3 \% \\
    \sigma_{\text{L2}} = 0,2 \% \\
    R_{16,2} = (416 \pm 25) \Omega\\
    L_{16,2} = (93 \pm 10) \cdot 10^{-3} H
  \end{align*}

  \begin{table}[h]
    \centering
    \caption{In der Tabelle sind die für die Rechnung notwendigen Widerstände für beide Widerstände angegeben.}
    \label{tab:Maxwell}
    \begin{tabular}{c c c c c c c}
      \toprule
      {Wert} & {$L_{\text{16}} [mH]$} & {$R_{\text{16}} [\Omega]$} & {$C_{\text{4}} [nF]$} & {$R_{\text{2}} [\Omega]$} & {$R_{\text{3}} [\Omega]$} & {$R_{\text{4}} [\Omega]$} \\
      \midrule 
      16 & 117 & 414 & 399 & 1000 & 293 & 707  \\
       & 74 & 417 & 399 & 332 & 557 & 443 \\
       & 102 & 417 & 399 & 664 & 386 & 614 \\
       & 83 & 414 & 450 & 332 & 555 & 445 \\
      \bottomrule
    \end{tabular}
  \end{table}

  \subsection{Wien-Robinson-Brücke}
  Bei der Wien-Robinson-Brücke soll die Abhängigkeit der Brückenspannung von der Frequenz betrachtet werden. Dazu wurde als Ausgangspunkt die Frequenz 
  bestimmt, bei der die Brückenspannung minimal wird.
  \begin{align*} 
    R' = 332 \Omega \: R = 1000 \Omega \: C = 660 nF \\
    \upnu = 241 Hz \: U_0 = 4 mV \: U_{\text{S}} =  500 mV
  \end{align*} 
  Von diesem Minimum aus wird die Frequenz variiert. Zum einem wird in Zehnerschritten 5 Schritte in beide Richtungen gegangen und zum anderen in 
  Hunderterschritten 5 Schritte in beide Richtungen. Daraus ergeben sich die in Tabelle \ref{tab:Wien} eingetragenen Werte. Die Frequenzabhängigkeit wird in
  Grafik \ref{fig:Graph} dargestellt. In diesem Plot wird das Verhältnis zwischen der Brückenspannung und der Speisespannung 
  $\frac{U_{\text{Br}}}{U_{\text{Speise}}}$ gegen das Verhältnis zwischen der Frequenz und der Frequenz bei minimaler Spannung $\frac{\upsilon}{\upsilon_0}$
  aufgetragen. Die berechnete Kurve und die Kurve der Messwerte weisen eine annähernd gleiche Form auf und auch die Minima stimmen überein. 
  \begin{table}[h]
    \centering
    \caption{In der Tabelle sind die Brückenspannungenspannungen und die zugehörigen Frequenzen angegeben.}
    \label{tab:Wien}
    \begin{tabular}{c c }
      \toprule
      {\text{Frequenz} [s]} & {U [\text{mV}]} \\
      \midrule 
        41 & 440    \\
        91  & 440      \\
        141 &  320 \\
        191 &  128 \\
        201 &  102 \\
        211 &  76 \\
        221 &  52,8 \\
        231 &  28,8 \\
        241 &  4 \\
        251 &  25,6 \\
        261 &  46,4 \\
        271 &  65,6 \\
        281 &  84,8 \\
        291 &  102 \\
        341 &  200 \\
        391 &  280 \\
        441 &  360 \\
        491 &  400 \\
        541 &  440 \\
        591 &  520 \\
      \bottomrule
    \end{tabular}
  \end{table}

  \begin{figure}[h]
    \centering
    \includegraphics[width=0.7\linewidth]{Brücke.pdf}
    \caption{ululul}
    \label{fig:Graph}
  \end{figure}

  \subsection{Klirrfaktor}
  Zuletzt soll noch der Klirrfaktor des Generators theoretisch berechnet werden. Dazu wird folgende Formel verwendet:
  \begin{align*}
    k = \frac{U_2}{U_1}
  \end{align*}
  $U_1$ entspricht dabei der Speisespannung und $U_2$ lässt sich wiefolgt berechnen:
  \begin{align*}
    
  \end{align*}

\end{document}