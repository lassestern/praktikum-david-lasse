\documentclass[titlepage = firstcover]{scrartcl}
\usepackage[aux]{rerunfilecheck}
\usepackage{fontspec}
\usepackage[main=ngerman, english, french]{babel}

% mehr Pakete hier
\usepackage{expl3}
\usepackage{xparse}

%Mathematik------------------------------------------------------
\usepackage{amsmath}   % unverzichtbare Mathe-Befehle
\usepackage{amssymb}   % viele Mathe-Symbole
\usepackage{mathtools} % Erweiterungen für amsmath
\usepackage[
  math-style=ISO,    % \
  bold-style=ISO,    % |
  sans-style=italic, % | ISO-Standard folgen
  nabla=upright,     % |
  partial=upright,   % /
]{unicode-math}% "Does exactly what it says on the tin."
\usepackage[section, below]{placeins}

% Laden von OTF-Mathefonts
% Ermöglich Unicode Eingabe von Zeichen: α statt \alpha

\setmathfont{Latin Modern Math}
%\setmathfont{Tex Gyre Pagella Math} % alternativ zu Latin Modern Math
\setmathfont{XITS Math}[range={scr, bfscr}]
\setmathfont{XITS Math}[range={cal, bfcal}, StylisticSet=1]

\AtBeginDocument{ % wird bei \begin{document}
  % werden sonst wieder von unicode-math überschrieben
  \RenewDocumentCommand \Re {} {\operatorname{Re}}
  \RenewDocumentCommand \Im {} {\operatorname{Im}}
}
\usepackage{mleftright}
\setlength{\delimitershortfall}{-1sp}

%Sprache----------------------------------------------------------
\usepackage{microtype}
\usepackage{xfrac}
\usepackage[autostyle]{csquotes}    % babel
\usepackage[unicode, pdfusetitle]{hyperref}
\usepackage{bookmark}
\usepackage[shortcuts]{extdash}
%Einstellungen hier, z.B. Fonts
\usepackage{booktabs} % Tabellen


\title{Brückenschaltungen}
\author{
  David Gutnikov\\
  \href{mailto:david.gutnikov@udo.edu}{david.gutnikov@udo.edu}
 \and 
  Lasse Sternemann\\
  \href{mailto:lasse.sternemann@udo.edu}{lasse.sternemann@udo.edu}
}
\date{Durchführung am 03.12.2019}

\begin{document}
    \maketitle
    \tableofcontents
    \newpage

    \section{Zielsetzung}
        Mithilfe von verschiedenen Brückenschaltungen sollen elektrische Widerstände und die damit verbundenen physikalischen Größen, wie z.B. die Kapazität
        eines Kondensators oder die Induktivität einer Spule, möglichst genau gemessen werden.



    \section{Theoretische Grundlagen}
        \subsection{Allgemeine Brückenschaltung}
          Brückenschaltungen bestehen aus vier oder mehr Widerständen, die wie in Abbildung \ref{fig:brueckenschalt} parallel und dann in Reihe geschaltet
          sind.
          \begin{figure}[h]
            \centering
            \caption{Die Abbildung einer einfachen Brückenschaltung.}
            \includegraphics[width = 0.4\linewidth]{Brueckenschaltung.png}
            \label{fig:brueckenschalt}
          \end{figure}
      
          \FloatBarrier
          \noindent
          Hierbei wird die Spannung zwischen den Punkten A und B, als Brückenspannung $U$ bezeichnet. Diese Spannung lässt sich mit Hilfe der Kirchhoffschen
          Gesetze durch die bekannten Größen der Widerstände und der Speisespannung $U_S$ darstellen.
          Die Kirchhoffschen Gesetze lauten:
          \subsubsection{Knotenregel}
            In jedem Knotenpunkt in dem sich elektrische Ströme verzweigen, ist die Summe der hineinfließenden und der herausfließenden Ströme gleich.
            \begin{equation}
              I_1 = I_2 + I_3
            \end{equation}
            \begin{figure}[h]
              \centering
              \caption{Das Beispiel eines Verzweigungspunktes el. Ströme.}
              \includegraphics[width = 0.4\linewidth]{Kirchhoff_1.png}
              \label{fig:kirchhoff1}
            \end{figure}
            \FloatBarrier

          \subsubsection{Maschenregel}
            Alle Teilspannungen einer Masche eines elektrischen Netzwerkes summieren sich zu null. Dabei legt man eine Zählrichtung fest (meistens im Uhrzeigersinn). Zeigt der Strom
            an den Teilspannungen in Zählrichtung, so wird diese positiv gezählt ($I_k > 0$), zeigt er in die entgegengesetzte Richtung, kriegt die Teilspannung ein negatives
            Vorzeichen ($I_k < 0$).
            \begin{equation}
              \sum_k E_k = \sum_k I_k \cdot R_k
            \end{equation}
            \begin{figure}[h]
              \centering
              \caption{Das Beispiel einer Masche in einem Stromnetzwerk.}
              \includegraphics[width = 0.4\linewidth]{Kirchhoff_2.png}
              \label{fig:kirchhoff2}
            \end{figure}
            \FloatBarrier

          Somit sähe die Brückenspannung wie folgt aus:
          \begin{equation}
            \label{eqn:brueckspannung}
            U = \frac{R_2 R_3 - R_1 R_4}{(R_3 + R_4) (R_1 + R_2)}U_S
          \end{equation}
          Da hier jedoch die Nullmethode angewendet wird muss $U = 0$ sein. Das wird durch folgendes Verhältnis erreicht wie in \eqref{eqn:brueckspannung}
          leicht zu erkennen ist:
          \begin{equation}
            \label{eqn:abgleichbed}
            R_1 R_4 = R_2 R_3
          \end{equation}
          Da diese Abgleichbedingung nur von den Verhältnissen der Widerstände abhängt wird diese Brückenschaltung "abgeglichene Brücke" genannt und es ist
          möglich eine Widerstandsmessung durchzuführen mit drei bekannten Widerständen.

        \subsection{Brückenschaltung mit komplexen Widerständen}
          Es ist auch möglich Kapazitäten und Iduktivitäten in eine Brückenschaltung mit einzubauen, dabei ist es sinnvoll kompelexe Widerstände zu benutzen,
          welche allgemein wie folgt aussehen:
          \begin{equation}
            \label{eqn:komplexwiderstand}
            R_Z = X + \text{i} Y
          \end{equation}
          Damit würde eine Brückenschaltung mit vier komplexen Widerständen folgende Abgleichbedingung nach \eqref{eqn:abgleichbed} haben:
          \begin{align}
            R_{Z1} R_{Z4} &= R_{Z2} R_{Z3} \\
            (X_1 + \text{i} Y_1)(X_4 + \text{i} Y_4) &= (X_2 + \text{i} Y_2)(X_3 + \text{i} Y_3)
          \end{align}
          Durch das Gleichstellen der Real- und Imaginärteile entstehen zwei Bedingungen:
          \begin{align}
            \label{eqn:komplexabgleich1}
            X_1 X_4 - Y_1 Y_4 &= X_2 X_3 - Y_2 Y_3 \\
            \label{eqn:komplexabgleich2}
            X_1 Y_4 + X_4 Y_1 &= X_2 Y_3 + X_3 Y_2
          \end{align}
          Beide Bedingungen müssen also bei einer Wechselstrombrücke erfüllt sein. Die Brückenspannung muss im Betrag verschwinden, dabei muss noch die Phase
          ausgeglichen werden. Deshalb muss jede Wechselstrombrücke zwei unabhängige, veränderliche Stellglieder haben.

        \subsection{Wheatstonesche Brücke}
          Mit der Wheatstoneschen Brücke werden Widerstände gemessen. Sie ist in Abbildung \ref{fig:wheatstone} dargestellt.
          \begin{figure}[h]
            \centering
            \caption{Die Wheatstonesche Brücke zur Widerstandsmessung.}
            \includegraphics[width = 0.4\linewidth]{Wheatstonesche Bruecke.png}
            \label{fig:wheatstone}
          \end{figure}
          Der gesuchte Widerstand $R_X$ wird durch Formel \eqref{eqn:abgleichbed} berechnet:
          \begin{equation}
            \label{eqn:abgleichbed}
            R_X = R_2 \frac{R_3}{R_4}
          \end{equation}

        \subsection{Kapazitätsmessbrücke}
          Da es keine idealen Kondensatoren in der Realität gibt, d.h. jeder reale Kondensator wandelt einen kleinen Teil der el. Energie in Wärme um,
          werden Kondensatoren durch Ersatzschaltbilder, als ein in Reihe geschalteter Widerstand R und idealer Kondensator dargestellt. Somit sieht der
          komplexe Widerstand des realen Kondesators wie folgt aus:
          \begin{equation}
            R_C = R - \text{i} \frac{1}{\omega C}
          \end{equation}
          Es gibt also zwei Unbekannte, den Widerstand $R_X$ und die Kapazität $C_X$. Deshalb werden auch zwei Abstimmffreiheitsgrade gebraucht, um der
          Phasenverschiebung aufgrund von $R_X$ entgegenzuwirken. Hier wird dazu der Widerstand $R_2$ verändert.
          \begin{figure}[h]
            \centering
            \caption{Die Kapazitätsmessbrücke.}
            \includegraphics[width = 0.4\linewidth]{Kapazitaetsmessbruecke.png}
            \label{fig:wheatstone}
          \end{figure}
          Es gelten für diese Schaltung nach \eqref{eqn:komplexwiderstand}
          \begin{align*}
            &X_1 = R_X, & X_2 &= R_2, & X_3 =& R_3, & X_4 =& R_4 \\
            &Y_1 = -\frac{1}{\omega C_X}, & Y_2 &= -\frac{1}{\omega C_2}, & Y_3 =& 0, & Y_4 =& 0
          \end{align*}
          Das wird in \eqref{eqn:komplexabgleich1} und \eqref{eqn:komplexabgleich2} eingesetzt und nach den gesuchten Größen umgestellt:
          \begin{align*}
            R_X = R_2 \frac{R_3}{R_4} &&\text{und}&& C_X = C_2 \frac{R_4}{R_3}            
          \end{align*}

          \subsection{Induktivitätsmessbrücke}
          Da es keine idealen Kondensatoren in der Realität gibt, d.h. jeder reale Kondensator wandelt einen kleinen Teil der el. Energie in Wärme um,
          werden Kondensatoren durch Ersatzschaltbilder, als ein in Reihe geschalteter Widerstand R und idealer Kondensator dargestellt. Somit sieht der
          komplexe Widerstand des realen Kondesators wie folgt aus:
          \begin{equation}
            R_C = R - \text{i} \frac{1}{\omega C}
          \end{equation}
          Es gibt also zwei Unbekannte, den Widerstand $R_X$ und die Kapazität $C_X$. Deshalb werden auch zwei Abstimmffreiheitsgrade gebraucht, um der
          Phasenverschiebung aufgrund von $R_X$ entgegenzuwirken. Hier wird dazu der Widerstand $R_2$ verändert.
          \begin{figure}[h]
            \centering
            \caption{Die Induktivitätsmessbrücke.}
            \includegraphics[width = 0.4\linewidth]{Induktivitaetsmessbruecke.png}
            \label{fig:wheatstone}
          \end{figure}
          Es gelten für diese Schaltung nach \eqref{eqn:komplexwiderstand}
          \begin{align*}
            &X_1 = R_X, & X_2 &= R_2, & X_3 =& R_3, & X_4 =& R_4 \\
            &Y_1 = -\frac{1}{\omega C_X}, & Y_2 &= -\frac{1}{\omega C_2}, & Y_3 =& 0, & Y_4 =& 0
          \end{align*}
          Das wird in \eqref{eqn:komplexabgleich1} und \eqref{eqn:komplexabgleich2} eingesetzt und nach den gesuchten Größen umgestellt:
          \begin{align*}
            R_X = R_2 \frac{R_3}{R_4} &&\text{und}&& C_X = C_2 \frac{R_4}{R_3}            
          \end{align*}

          \section{Versuchsdurchführung}
        Das konstante Volumen einer Körperprobe ist bei Durchführung eines Versuches schwierig zu erhalten.
        Deshalb arbeitet man hier mit einer anderen konstanten Größe, dem konstanten Druck, welcher hier
        der Atmosphärendruck ist. Es werden Körper aus verschiedenen Materialien der Massen $m_\text{k}$ in Wasser
        eingetaucht und das Wasser wird auf die Temperatur $T_\text{k}$ erhitzt. Dabei nimmt der Körper die Temperatur $T_\text{k}$ an. Ab einer genügend hohen Temperatur
        wird der Körper aus dem Gefäß entfernt und in ein mit Wasser mit Zimmertemperatur $T_\text{w}$ gefülltes
        Dewar-Gefäß getaucht. Nach ca. 2 Minuten des Wartens ergibt sich eine Mischtemperatur $T_\text{m}$ im Gefäß.
        Die verschiedenen Temperaturen werden mit einem Thermometer gemessen.
        Die spezifischen Wärmekapazitäten werden, dann mit der spezifischen Wärmekapazität des Körpers $c_\text{k}$, des Wassers im Dewar-Gefäß bei ca. Zimmertemperatur
        $c_\text{w}$, der Wärmekapazität des Dewar-Gefäßes $c_\text{g}m_\text{g}$ und der Masse des Wasser im Dewar-Gefäß
        $m_\text{w}$ nach Formel \eqref{eqn:spezWärmekapazität} bestimmt.

        Um die Wärmekapazität des Dewar-Gefäßes zu bestimmen ist ein weiteres Experiment nötig. Anstatt einer Körperprobe
        wird nun Wasser der Masse $m_\text{y}$ auf eine Temperatur $T_\text{y}$ erhitzt und es wird damit genauso verfahren wie mit den Körpern.
        D.h. es wird in das Dewar-Gefäß gegossen, worin sich schon Wasser der Masse $m_\text{x}$ und der Temperatur
        $T_\text{x}$ befindet und mit welchem sich eine Mischtemperatur $T_\text{m}^{'}$ ergibt. Analog zu Formel
        \eqref{eqn:spezWärmekapazität} gilt für die Wärmekapazität des Dewar-Gefäßesnach Formel \eqref{eqn:Wärmekapazität}.

\end{document}