
\documentclass[titlepage = firstcover]{scrartcl}
\usepackage[aux]{rerunfilecheck}
\usepackage{fontspec}
\usepackage[main=ngerman, english, french]{babel}

% mehr Pakete hier
\usepackage{expl3}
\usepackage{xparse}

%Mathematik------------------------------------------------------
\usepackage[nice]{nicefrac}
\usepackage{amsmath}   % unverzichtbare Mathe-Befehle
\usepackage{amssymb}   % viele Mathe-Symbole
\usepackage{mathtools} % Erweiterungen für amsmath
\usepackage[
  math-style=ISO,    % \
  bold-style=ISO,    % |
  sans-style=italic, % | ISO-Standard folgen
  nabla=upright,     % |
  partial=upright,   % /
]{unicode-math}% "Does exactly what it says on the tin."
\usepackage[section, below]{placeins}

% Laden von OTF-Mathefonts
% Ermöglich Unicode Eingabe von Zeichen: α statt \alpha

\setmathfont{Latin Modern Math}
%\setmathfont{Tex Gyre Pagella Math} % alternativ zu Latin Modern Math
\setmathfont{XITS Math}[range={scr, bfscr}]
\setmathfont{XITS Math}[range={cal, bfcal}, StylisticSet=1]

\AtBeginDocument{ % wird bei \begin{document}
  % werden sonst wieder von unicode-math überschrieben
  \RenewDocumentCommand \Re {} {\operatorname{Re}}
  \RenewDocumentCommand \Im {} {\operatorname{Im}}
}
\usepackage{mleftright}
\setlength{\delimitershortfall}{-1sp}
\usepackage[version=4]{mhchem}

%Sprache----------------------------------------------------------
\usepackage{microtype}
\usepackage{xfrac}
\usepackage[autostyle]{csquotes}    % babel
\usepackage[german, unicode, pdfusetitle]{hyperref}
\usepackage{bookmark}
\usepackage[shortcuts]{extdash}
%Einstellungen hier, z.B. Fonts
\usepackage{booktabs} % Tabellen

\setlength{\parindent}{0pt}

\title{V702 - Aktivierung mit Neutronen}
\author{
  David Gutnikov\\
  \href{mailto:david.gutnikov@udo.edu}{david.gutnikov@udo.edu}\\
  Lasse Sternemann\\
  \href{mailto:lasse.sternemann@udo.edu}{lasse.sternemann@udo.edu}
}
\date{Bearbeitet am 29.05.2020}

\begin{document}
    \maketitle
    \newpage
    \tableofcontents
    \newpage


    \section{Theoretische Grundlagen}
        \subsection{Erzeugung instabiler Kerne mit Neutronen}
            Um Zefallsprozesse hervorzurufen, werden Atome mit Neutronen beschossen. Dabei dringt das Neutron, ohne die Coulomb-Barriere überwinden zu müssen, in den Kern des beschossenen Atoms ein. 
            Der um ein Neutron erweiterter Kern $A^*$ heißt Zwischenkern und ist um die kinetische- sowie Ruheenergie des Neutrons energiereicher im Vergleich zu seinem Vorgänger. Der nun angeregte Kern
            gibt die übrige Energie über ein Photon ab und befindet sich nun wieder im Grundzustand $A$.
            
            \begin{equation*}
                \ce{^{m}_{z}A} + \ce{^{1}_{0}n} \longrightarrow \ce{^{m+1}_{z}A}^* \longrightarrow \ce{^{m+1}_{z}A} + \gamma
            \end{equation*}

            \noindent
            Der so neu entstandene Kern ist aufgrund der höheren Neutronenzahl häufig instabil und wird daher zerfallen. Dies geschieht durch den $\beta^-$-Zerfall, bei dem der instabile Kern in einen
            stabilen Kern übergeht und dabei ein Elektron e, ein Antielektronneutrino $\bar{\nu_e}$ sowie deren kinetische Energie freigesetzt wird. Die kinetische Energie entspringt der Massendifferenz
            (Massendeffekt) beim $\beta^-$-Zerfall mit $\Delta E = \Delta mc^2$.

            \begin{equation*}
                \ce{^{m+1}_{z}A} \longrightarrow \ce{^{m+1}_{z+1}C} + \text{e} + \text{E}_{\text{Kin}} + \bar{\nu_e}
            \end{equation*}

        \subsection{Aufnahme von Neutronen}
            Die Wahrscheinlichkeit, dass ein Neutron in einen Kern eindringt wird durch den Wirkungsquerschnitt quantifiziert, dessen Einheit dem barn entspricht ($1 \text{barn} \equiv 10^{-24} cm^2$)
            und der bei einer 1 $cm^2$ großen Folie über folgende Formel berechnet wird
            
            \begin{equation*}
                \sigma = \frac{\text{u}}{\text{nKd}}
            \end{equation*}

            \noindent
            Dabei steht u für die Anzahl an Neutroneneinfängen, n für die Anzahl an einfallendne Neutronen, d für die Dicke des Schirmmaterials und K für die Anzahl der Atome pro Volumen in $cm^{-3}$.
            Da der Wirkungsquerschnitt für das Eindringen eines Neutrons geschwindigkeitsabhängig ist, wird bei der Berechnung zwischen schnellen und langsammen Neutronen unterschieden. Diese 
            Unterscheidung erfolgt über die ebenfalls geschwindigkeitsabhängige De-Broglie-Wellenlänge:

            \begin{equation*}
                \lambda = \frac{\text{h}}{m_{\text{Neutron}} \cdot v_{\text{Neutron}}}
            \end{equation*}

            \noindent

            \subsubsection*{Schnelle Neutronen}
                Es handelt sich um schnelle Elektronen, wenn deren Geschwindigkeit so groß ist, dass deren De-Broglie-Wellenlänge gegenüber dem Kernradius ($\approx 10^{-12}$cm) klein wird. Dabei 
                lässt sich das System analog zur Streuung von Licht an einem makroskopischen Objekt betrachten.

            \subsubsection*{Langsame Neutronen}
                Interessanter ist der Fall langsammer Neutronen, bei denen die De-Broglie-Wellenlänge groß gegen den Kernradius ist. Hierbei kann der Wirkungsquerschnitt in Abhängigkeit von der 
                der Energie des Neutrons und den Energieniveaus des Zwischenkerns berechnen. $\widetilde{c}$ und $\sigma_0$ sind dabei Konstanten der zugehörigen Kernreaktion.

                \begin{align*}
                    \sigma(\text{E}) &= \sigma_0 \cdot \sqrt{\frac{\text{E}_{\text{r}_i}}{\text{E}}} \cdot \frac{\widetilde{c}}{\left(\text{E} - \text{E}_{\text{r}_i}\right)^2 + \widetilde{c}}\\
                    \text{E} &= \frac{1}{2} \cdot m_{\text{Neutron}} \cdot v_{\text{Neutron}}^2
                \end{align*}

                \noindent
                Wenn nun die Energie des einfallenden Teilchens viel kleiner ist als die des jeweiligen Energieniveaus des Kerns ist der Wirkungsquerschnitt proportional zum Kehrwert der Wurzel der
                Neutronenenergie und somit zum Kehrwert der Geschwindigkeit des Neutrons. Da zur Aktivierung der Kerne ein möglichst hoher Wirkungsquerschnitt gewünscht ist, werden langsamme und
                niederenergetische Neutronen bevorzugt.

        
        \subsection{Erzeugung niederenergetischer Neutronen}
            Neutronen sind ungebunden instabil und kommen daher nicht natürlich im freiem Zustand vor. Die für den Neutroneneinfang vorgesehenen Neutronen werden daher durch den Beschuss von 
            Beryllium mit Alpha-Teilchen erzeugt.

            \begin{equation*}
                \ce{^{9}_{4}Be} + \ce{^{4}_{2}He} \longrightarrow \ce{^{12}_{6}C} + \ce{^{1}_{0}n}
            \end{equation*}

            \noindent
            Die dabei freigesetzten Neutronen haben jedoch eine kontinuierliche Energieverteilung mit bis zu 13,7 MeV und sind damit nicht niederenergetisch. Daher wird die Quelle von einer 
            Materieschicht umhüllt. Wenn die Neutronen in diese eintreten kommt es solange zu elastischen Stößen bis die kinetische Energie der der umgebenden Moleküle entspricht. Dann beträgt
            die Neutronenenergie, der nun als thermische Neutronen bezeichneten Neutronen, etwa 0,025 eV, was einer Temperatur der Teilchenmenge von 290 Kelvin entspricht. Dieser Prozess läuft 
            umso schneller ab, desto ähnlicher sich die Massen der Stoßpartner sind. Dies lässt sich aus Gleichung \ref{eqn:Stoß} lesen. Diese gibt die pro Stoß übergebende Energie in Abhängigkeit
            von der Anfangsenergie $\text{E}_0$ und den Massen der verschiedenen Stoßpartner M und m an. Die kleinste Masse wäre bei Wasserstoff gegeben, sodass als Stoßpartner die Neutronen von
            Paraffin, das hauptsächlich aus Wasserstoff besteht, genutzt werden. 

            \begin{equation}
                \text{E}_{\text{ü}} = \text{E}_0 \cdot \frac{4\text{Mm}}{\left(M + m\right)^2}
                \label{eqn:Stoß}
            \end{equation}

        \subsection{Zerfallsgesetz}
            Gemäß dem Zerfallsgesetz radioaktiver Isotope ist die Anzahl der nach einer gewissen Zeit t von der Ausgangsanzahl $\text{N}_0$ an vorhandenen Teilchen noch übrige Teilchenzahl N
            über folgende Formel berechenbar. Dabei steht $\lambda$ für die Zerfallskonstante.
            
            \begin{equation}
                \text{N}(t) = \text{N}_0 \cdot e^{-\lambda t}
                \label{eqn:Zerfallsgesetz}
            \end{equation}

            \noindent
            Die Halbwertszeit beschreibt die Dauer in der von der anfänglichen Teilchenzahl nur noch die Hälfte übrig ist. Dies lässt sich wie folgt ausdrücken.

            \begin{equation}
                \frac{1}{2} \cdot \text{N}_0 = \text{N}_0 \cdot e^{-\lambda T_{\frac{1}{2}}} \qquad \longrightarrow \qquad T_{\nicefrac{1}{2}} = \frac{\ln(2)}{\lambda}
                \label{eqn:Halbwertszeit}
            \end{equation}

            \noindent
            So kann die spezifische Halbwertszeit für die Isotope bestimmt werden.
            Da die Anzahl der vorhandenen Kerne jedoch nur umständlich zu bestimmen ist, wird die Zerfallskonstante stattdessen über die Anzahl der zerfallenden Kerne pro Zeitintervall $N_{\Delta t}$
            bestimmt, da diese Zerfälle einfacher detektiert werden können. Die Anzahl der zerfallenden Kerne ergibt sich durch Subtraktion der nach einem Intervall $\Delta t$ vorhandenen Kernen von
            den vor diesem Intervall vorhandenen Kernen.

            \begin{equation*}
                \text{N}_{\Delta t}(t) = \text{N}(t) - \text{N}(t + \Delta t)
            \end{equation*}

            \noindent
            Anwenden des Zerfallsgesetzes \ref{eqn:Zerfallsgesetz} und anschließendes anwenden des natürlichen Logarithmus liefert eine weitere Möglichkeit zur Bestimmung der Halbwertszeit.
            
            \begin{equation}
                \text{N}_{\Delta t}(t) = \text{N}_0 \cdot e^{-\lambda t} - \text{N}_0 \cdot e^{-\lambda (t + \Delta t)} = \text{N}_0 \cdot \left(1 - e^{-\lambda \Delta t}\right) \cdot e^{-\lambda t}
                \label{eqn:Zerfallsgesetz_}
            \end{equation}
                
            \begin{equation}
                \ln\left(\text{N}_{\Delta t}(t)\right) = \ln \left(\text{N}_0 \cdot \left(1 - e^{-\lambda \Delta t}\right)\right) - \lambda t
                \label{eqn:theorieGerade}
            \end{equation}
            
            \noindent
            Auf der echten Seite dieser Gleichung ist allein t variabel und auf der linken Seite $N_{\Delta t}$, sodass die Zerfallskonstante über eine lineare Regression bestimmbar ist und daraus 
            wiederum die Halbwertszeit berechnet werden kann. Dabei muss
            $\Delta t$ exakt gewählt sein, da zu kleine Werte zu große statistische Fehler mit sich bringen und zu große Zeiten, die gegen die Halbwertszeit laufen einen systematoschen Fehler mit
            sich bringen.

        \subsection{Besonderheiten des Experiments}
            Die Bestimmung der Halbwertszeit soll für zwei Elemente durchgeführt werden.

                \subsubsection*{Vanadium}
                Das durch Neutroneneinfang aktivierte Vanadium zefällt wiefolgt:

                \begin{equation*}
                    \ce{^{51}_{23}V} + \ce{^{1}_{0}n} \longrightarrow \ce{^{52}_{23}V}  \longrightarrow \ce{^{52}_{24}Cr} + \text{e} + \bar{\nu_e}
                \end{equation*}

                \noindent 
                Es entsteht nur ein instabiles Isotop und die Bestimmung über eine Ausgleichsgerade lässt sich problemlos durchführen.
                
                \subsubsection*{Rhodium}
                Im Gegensatz zum Vanadium entstehen bei Rhodium durch Neutroneneinfang zwei instabile Isotope mit verschiedener Wahrscheinlichkeit.

                \begin{equation}
                    \ce{^{103}_{45}Rh} + \ce{^{1}_{0}n}  \begin{cases}
                        \xrightarrow{10\%} \ce{^{104i}_{45}Rh} \longrightarrow \ce{^{104}_{45}Rh} + \gamma \longrightarrow \ce{^{104}_{46}Pd} + \text{e} + \bar{\nu_e} \\
                        \xrightarrow{90\%} \ce{^{104}_{45}Rh} \longrightarrow \ce{^{104}_{46}Pd} + \text{e} + \bar{\nu_e}
                    \end{cases}
                    \label{eqn:Rhodium}
                \end{equation}

                \noindent
                Diese Isotope zerfallen auch beide in andere Elemente. Gemessenen werden kann jedoch nur die Gesamtaktivität der beiden Zerfallsketten. Da die beiden Isotope unterschiedlich schnell 
                zerfallen ist nach einer gewissen Zeit $t^*$ nur noch das langlebige Isotop für die Aktivität verantwortlich. So kann für den Graphen der Messwerte eine lineare Regression für den 
                Bereich $t>t^*$ angefertigt und damit die Zerfalsskonstante des langlebigen Isotops bestimmt werden. Durch Abziehen diesen Untergrunds von dem Messwerten und anschließender linearen 
                Regression über den Bereich $t<t^*$ kann dann die Zerfallskonstante des kurzlebigen Isotops bestimmt werden. Die beim Zefall von $\ce{^{104i}_{45}Rh}$ frei werdende Gamma-Strahlung
                muss nicht berücksichtigt werden, da sie nur geringfügig zur gesamten Zählung beiträgt. 

                \FloatBarrier

                \begin{figure}[h]
                  \centering
                  \includegraphics[width = 0.6\textwidth]{Bilder/RhSchema.png}
                  \caption{In der Abbildung ist eine schematische Darstellung der Zerfallskurve zweier Isotope, deren Halbwertszeiten stark verschieden sind. Die Anzahl der zerfallenden Kerne ist dabei logarithmisch gegen die Zeit aufgetragen. Die Zerfallskurve eines einzelnen Isotops würde linear verlaufen und das tun auch die linearen Regressionen der einzelnen Isotope (gestrichelt)  [1]}
                  \label{fig:RhSchema}
                \end{figure}

                \FloatBarrier

    \newpage
    \section{Durchführung}
        \subsection{Versuchsaufbau}

            \FloatBarrier

            \begin{figure}[h]
              \centering
              \includegraphics[width = 0.6\textwidth]{Bilder/Aufbau.png}
              \caption{In der Abbildung ist eine schematische Darstellung der Zerfallskurve zweier Isotope, deren Halbwertszeiten stark verschieden sind. Die Anzahl der zerfallenden Kerne ist dabei logarithmisch gegen die Zeit aufgetragen. Die Zerfallskurve eines einzelnen Isotops würde linear verlaufen und das tun auch die linearen Regressionen der einzelnen Isotope (gestrichelt)  [1]}
              \label{fig:Aufbau}
            \end{figure}

            \FloatBarrier

            \noindent
            Wie im schematischen Aufbau zu erkennen, ist die Probe in einen Bleimantel gehüllt, um den Einfluss äußerer Strahlung auf die Messung zu verringern. Bei jedem zerfallenden Kern wird ein
            $\beta^-$-Teilchen freigesetzt, das auf dem Geiger-Müller-Zählrohr einen elektrischen Impuls erzeugt, der verstärkt und dann gemessenen wird. Ein Zeitschalter wechselt nach verstreichen
            des vorgewählten Zeitintervalls $\Delta t$ auf einen anderen Zähler, sodass ohne Unterbrechng zur Notierung der Daten weitergemessen werden kann.

        \subsection{Untergrundbestimmung}
            Trotz der Bleiummantelung ist ein Anteil an äußerer Strahlung vorhanden. Um diesen Huntergrund später abzuziehen, wird dieser zunächst bestimmt. Dazu werden Messungen ohne Probe im 
            Zählrohr mit einer Integrationszeit von 300 Sekunden durchgeführt.

        \subsection{Halbwertszeitbestimmung von Vanadium}
            Zur Halbwertszeitbestimmung von Vanadium wird eine Vanadiumprobe zunächst an einer Neutronenquelle platziert. Nachdem diese dort aktiviert wurde, wird sie in dem Zählrohr platziert und 
            die Aktivität über ein Messintervall von 30 Sekunden gemessen. Alle 30 Sekunden wechselt der Zeitschalter auf den anderen der zwei Zähler und ermöglicht so ein Ablesen der Zählrate auf
            dem anderen Zähler.

        \subsection{Halbwertszeitbestimmung von Rhodium}
            Die Rhodiumprobe wird entsprechend der Vorbereitung der Vanadiumprobe auch neben der Neutronenquelle aktiviert. Die Messung der Aktivität verläuft analog zur Messung der Vanadiumprobe
            mit einer Messzeit von 15 Sekunden. Auch hier ermöglicht der Zeitschalter ein kontinuierliches Messen.
            
    \section{Auswertung}
        \subsection{Untergrundzählrate}
            Zuerst wird die Untergrundzählrate, die von jeder Messung abgezogen werden muss, um korrekte Messwerte zu bekommen.
            Deshalb wurden Werte für die Untergrundrate jeweils mit einer Integrationszeit von 300s gemessen:
            \begin{equation*}
                N_U = \{129, 143, 144, 136, 139, 126, 158\}
            \end{equation*}

            Der Mittelwert dieser Werte ist $N_{\text{U,m}} = (139 \pm 10)\,$Imp/300s. Da die Zählraten der Proben jedoch mit verschiedenen Integrationszeiten gemessen wurden, muss der Mittelwert auf diese Zeiten angepasst werden.

        \subsection{Vanadium}
            Da bei dieser Messreihe eine Messzeit von $\Delta t = 30$s verwendet wurde, wird die Untergrundzählrate durch 10 geteilt und von den gemessenen Werten $N^*_{\Delta t}$ abgezogen:
            \begin{equation*}
                N_{\Delta t} = N^*_{\Delta t} - \frac{N_{\text{U,m}}}{10}
            \end{equation*}
            
            Die Messwerte werden halblogarithisch aufgetragen. Nach \autoref{eqn:theorieGerade} sollten die Werte theoretisch ca. proportional zur Zerfallskonstante $\lambda$ sein:
            \begin{align*}
                &\underbrace{\ln\left(N_{\Delta t}(t)\right)} = \underbrace{\ln \left(N_0 \cdot \left(1 - e^{-\lambda \Delta t}\right)\right)} \underbrace{ - \lambda} \hspace{2pt} \cdot \hspace{3pt} \underbrace{t} \\
                & \hspace{26pt} y \hspace{25pt} = \hspace{47pt} b \hspace{49pt} + m \cdot x
            \end{align*}            

            Deshalb kann eine lineare Regression durch alle Werte durchgeführt werden, um die Zerfallskonstante und damit auch die Halbwertszeit anhand von \autoref{eqn:Halbwertszeit} zu berechnen.
            Es ist jedoch leicht in \autoref{fig:Vanadium} zu erkennen, dass die Messwerte aufgrund der halblogarithmischen Darstellungsweise stark divergieren.

            Deswegen wird zuerst mithilfe der ersten linearen Regression die vorläufige Halbwertszeit $T_{\nicefrac{1}{2}}$ bestimmt und für die zweite lineare Regression werden nur die Messwerte bis ca. zur doppelten Halbwertszeit, in diesem Fall die ersten 14, verwendet.

            \begin{figure}[h]
                \centering
                \includegraphics[width = 0.8\textwidth]{Bilder/HalbwertszeitGraph_Vanadium.png}
                \caption{Es sind die Werte aus \autoref{tab:Vanadium} halblogarithmisch aufgetragen und eine Ausgleichgerade wird zuerst durch alle Punkte und eine zweite Ausgleichsgerade durch die ersten 14 Werte durchgeführt.}
                \label{fig:Vanadium}
            \end{figure}
  
            \FloatBarrier

            Für die erste und zweite Ausgleichsgerade ergeben sich folgende Werte:
            \begin{align*}
                m_1 = (-3,17 \pm 0,16 \cdot 10^{-3}) \text{s}^{-1} \hspace{75pt} b_1 = (5,19 \pm 0,13) \\
                m_2 = (-3,23 \pm 0,24 \cdot 10^{-3}) \text{s}^{-1} \hspace{75pt} b_2 = (5,32 \pm 0,06)
            \end{align*}

            Also ergibt sich mit \autoref{eqn:Halbwertszeit} für die Halbwertszeiten
            \begin{align*}
                T_{\nicefrac{1}{2} \text{, 1}} = (219 \pm 11)\, \text{s} \\
                T_{\nicefrac{1}{2} \text{, 2}} = (214 \pm 16)\, \text{s}
            \end{align*}
            und aus dem y-Achsenabschnitt werden die Vorfaktoren
            \begin{align*}
                \text{e}^{b_1} = N_0 \left(1 - e^{-\lambda_1 \Delta t}\right) = (178 \pm 23) \,\text{Imp} \\
                \text{e}^{b_2} = N_0 \left(1 - e^{-\lambda_2 \Delta t}\right) = (205 \pm 12) \,\text{Imp}
            \end{align*}
            berechnet.

            Die Abweichung der Halbwertszeit vom Literaturwert $T_{\nicefrac{1}{2} \text{,lit}} = 224,6 \,$s sind:
            \begin{align*}
                a_1 = \frac{T_{\nicefrac{1}{2} \text{,lit}} - T_{\nicefrac{1}{2} \text{, 1}}}{T_{\nicefrac{1}{2} \text{,lit}}} = 2,6 \, \% \\
                a_2 = \frac{T_{\nicefrac{1}{2} \text{,lit}} - T_{\nicefrac{1}{2} \text{, 2}}}{T_{\nicefrac{1}{2} \text{,lit}}} = 4,5 \, \%
            \end{align*}
        
        \newpage
        \subsection{Rhodium}
            Bei dem Zerfall von Rhodium besteht das Problem, dass gleichzeitig zwei verschiedene Zerfallsprozesse stattfinden (siehe \autoref{eqn:Rhodium}).

            Davon läuft der Eine langsamer und der Andere schneller ab. Deshalb sind nach einer gewissen Zeit $t^*$ in der Messung alle schnellen Zerfallsprozesse abgelaufen und es bleibt nur noch der langsame Zerfall. Das kann ist gut in \autoref{fig:Rhodium} daran erkennbar, dass der Graph zuerst schnell fällt und dann ab ca. $t^*$ einen linearen Anteil aufweist.

            Es wird also zuerst die lineare Regression durch den linearen Teil des Graphen gemacht. Bei diesem Fall wurden die 18 letzten Messwerte benutzt.

            Dann wird die Ausgleichsgerade extrapoliert und von dem Graphen subtrahiert. Dabei ist darauf zu achten, dass alle Werte der Ausgleichsgeraden unterhalb der Werten $N_{\Delta t}(t)$ mit $t < t^*$ liegen.
            \begin{figure}[h]
                \centering
                \includegraphics[width = 0.8\textwidth]{Bilder/HalbwertszeitGraph_Rhodium.png}
                \caption{Es sind die Werte aus \autoref{tab:Rhodium} halblogarithmisch aufgetragen und eine Ausgleichgerade durch die letzten 18 Werte durchgeführt. Anschließend wird eine Ausgleichsgerade durch die ersten 14 korrigierten Werte durchgeführt.}
                \label{fig:Rhodium}
            \end{figure}

              \FloatBarrier
              
              Aus den Parametern der Ausgleichsgerade für den langsamen Zerfall
              \begin{equation*}
                  m_l = (-3,31 \pm 1,00 \cdot 10^{-3}) \text{s}^{-1} \hspace{75pt} b_l = (4,57 \pm 0,54)
              \end{equation*}

              ergibt sich mit der gleichen Rechnung wie beim Vanadium die Halbwertszeit $T_{\nicefrac{1}{2} \text{, l}} = (209 \pm 63) \,$s und der Vorfaktor $e^{b_l} = (96,5 \pm 51,8) \,$Imp. \\

              Wie in \autoref{fig:Rhodium} zu erkennen ist, macht es keinen Sinn die Ausgleichsgerade in einen Graphen einzuzeichnen, in dem der langsame Zerfall noch vorhanden ist.
              \begin{figure}[h]
                \centering
                \includegraphics[width = 0.8\textwidth]{Bilder/HalbwertszeitGraph_Rhodium_kurzlebig.png}
                \caption{Es sind die ersten 14 Werte aus \autoref{tab:Rhodium} halblogarithmisch aufgetragen, nachdem der langsame Zerfall abgezogen wurde.}
                \label{fig:RhodiumSchnell}
            \end{figure}

            \FloatBarrier

            Aus den Parametern der Ausgleichsgerade für den schnellen Zerfall
            \begin{equation*}
                m_s = (-18,1 \pm 0,4 \cdot 10^{-3}) \text{s}^{-1} \hspace{75pt} b_s = (6,76 \pm 0,05)
            \end{equation*}

            ergibt sich mit der gleichen Rechnung wie beim Vanadium die Halbwertszeit $T_{\nicefrac{1}{2} \text{, s}} = (38,4 \pm 0,8)$ und der Vorfaktor $e^{b_s} = (862,8 \pm 41) \,$Imp. \\

            Die Abweichung der Halbwertszeit des langsamen und schnellen Zerfalls von den Literaturwerten $T_{\nicefrac{1}{2} \text{,l,lit}} = 260,4 \,$s und $T_{\nicefrac{1}{2} \text{,s,lit}} = 42,3 \,$s sind jeweils:
            \begin{align*}
                &a_l = 19,6 \, \% \\
                &a_s = 9,2 \, \%
            \end{align*}

            Die Theoriekurve in \autoref{fig:Rhodium} wurde mit den jetzt bekannten Werten für die Zerfallskonstanten gezeichnet, anhand folgender Formel, die aus der Summe der beiden Gleichungen \ref{eqn:Zerfallsgesetz_} für den langsamen Zerfall und schnellen Zerfall entsteht:
            \begin{align*}
                &&N_{\Delta t, ges} &= N_0 (1 - e^{-\lambda_l \Delta t}) \cdot e^{-\lambda_l t} + N_0 (1 - e^{-\lambda_s \Delta t}) \cdot e^{-\lambda_s t} \\
                \implies &&\ln\Bigl(N_{\Delta t, ges} \Bigr) &= \ln\Bigl(N_0 (1 - e^{-\lambda_l \Delta t}) \cdot e^{-\lambda_l t} + N_0 (1 - e^{-\lambda_s \Delta t}) \cdot e^{-\lambda_s t} \Bigr)
            \end{align*}
      
    \newpage
    \section{Diskussion}
        Die Abweichungen der gemessenen Halbwertszeiten von den Literaturwerten sind beim Vanadium relativ gering.
        Allerdings ist die Abweichung der Halbwertszeit, die mit der vermeintlich verbesserten Rechnung bestimmt wurde größer als die die mit der Ausgleichsrechnung über alle Werte bestimmt wurde.

        Das liegt nur daran, dass die die Messwerte zufällig so verteilt sind, dass sich so eine Steigung ergibt. Normalerweise würde die zweite Ausgleichsrechnung ein Resultat ergeben, dass näher am tatsächlichen Wert liegt als die Erste. \\

        Beim Rhodium sind die Abweichungen der Halbwertszeit des langsamen Zerfalls natürlicherweise größer, da $t^*$ zum Einen nur per Augenmaß gewählt wird und die Werte in der zweiten Hälfte des Graphen relativ weit gestreut sind.

        Da die Werte für den schnellen Zerfall durch Subtraktion der Werte für den langsamen Zerfall entstehen, fließt die Ungenauigkeit der Ausgleichsrechnung für den langsamen Zerfall auch in die Berechnung der Halbwertszeit des schnellen Zerfalls mit ein und die Abweichung ist auch hier im Vergleich zum Vanadium relativ groß.



    
    \newpage
    \section{Daten}
        \begin{table}[h]
            \centering
            \caption{Hier sind die Messwerte der Zählrate des Zerfalls von Vanadium mit einem Integrationszeitraum von 30s angegeben.}
            \label{tab:Vanadium}
            \begin{tabular}{c c c c}
            \toprule
            {$t$ [s]} & {$N$ [Impulse/30s]} & {$t$ [s]} & {$N$ [Impulse/30s]} \\
            \midrule
            30	 &  189  &  690   &  35 \\
            60	 &  197  &  720   &  19 \\
            90	 &  150  &  750   &  28 \\
            120	 &  159  &  780   &  27 \\
            150	 &  155  &  810   &  36 \\
            180	 &  132  &  840   &  25 \\
            210	 &  117  &  870   &  29 \\
            240	 &  107  &  900   &  18 \\
            270	 &  94   &  930   &  17 \\
            300	 &  100  &  960   &  24 \\
            330	 &  79   &  990   &  21 \\
            360	 &  69   &  1020  &  25 \\
            390	 &  81   &  1050  &  21 \\
            420	 &  46   &  1080  &  24 \\
            450	 &  49   &  1110  &  25 \\
            480	 &  61   &  1140  &  17 \\
            510	 &  56   &  1170  &  20 \\
            540	 &  40   &  1200  &  19 \\
            570	 &  45   &  1230  &  20 \\
            600	 &  32   &  1260  &  18 \\
            630	 &  27   &  1290  &  16 \\
            660	 &  43   &  1320  &  17 \\                        
            \bottomrule
            \end{tabular}
        \end{table}

        \FloatBarrier

        \begin{table}[h]
            \centering
            \caption{Hier sind die Messwerte der Zählrate des Zerfalls von Rhodium mit einem Integrationszeitraum von 15s angegeben.}
            \label{tab:Rhodium}
            \begin{tabular}{c c c c}
            \toprule
            {$t$ [s]} & {$N$ [Impulse/15s]} & {$t$ [s]} & {$N$ [Impulse/15s]} \\
            \midrule
                15    &   667   &   345   &   36 \\ 
                30    &   585   &   360   &   38 \\ 
                45    &   474   &   375   &   34 \\
                60    &   399   &   390   &   40 \\
                75    &   304   &   405   &   21 \\
                90    &   253   &   420   &   35 \\
                105   &   213   &   435   &   33 \\
                120   &   173   &   450   &   36 \\
                135   &   152   &   465   &   20 \\
                150   &   126   &   480   &   24 \\
                165   &   111   &   495   &   30 \\
                180   &    92   &   510   &   30 \\
                195   &    79   &   525   &   26 \\
                210   &    74   &   540   &   28 \\
                225   &    60   &   555   &   23 \\
                240   &    52   &   570   &   20 \\
                255   &    56   &   585   &   28 \\
                270   &    53   &   600   &   17 \\
                285   &    41   &   615   &   26 \\
                300   &    36   &   630   &   19 \\
                315   &    37   &   645   &   13 \\
                330   &    32   &   660   &   17 \\
                \bottomrule
            \end{tabular}
        \end{table}

        \FloatBarrier

    \newpage
    \section{Literatur}
        [1] \textit{Versuchsanleitung V702 - Aktivierung der Neutronen.} TU Dortmund, 2020 \newline
        [2] \textit{Nuclear Structure and Decay Data}, Brookhaven National Laboratory \newline
        \url{https://www.nndc.bnl.gov/nudat2/}

\end{document}
